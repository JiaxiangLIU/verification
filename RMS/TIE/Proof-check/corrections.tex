%\documentclass[a4paper]{article}
\documentclass[12pt,onecolumn]{IEEEtranTIE}
%
% If IEEEtran.cls has not been installed into the LaTeX system files,
% manually specify the path to it like:
% \documentclass[10pt,journal,compsoc]{../sty/IEEEtran}


% Some very useful LaTeX packages include:
% (uncomment the ones you want to load)


% *** MISC UTILITY PACKAGES ***
%
%\usepackage{ifpdf}
% Heiko Oberdiek's ifpdf.sty is very useful if you need conditional
% compilation based on whether the output is pdf or dvi.
% usage:
% \ifpdf
%   % pdf code
% \else
%   % dvi code
% \fi
% The latest version of ifpdf.sty can be obtained from:
% http://www.ctan.org/tex-archive/macros/latex/contrib/oberdiek/
% Also, note that IEEEtran.cls V1.7 and later provides a builtin
% \ifCLASSINFOpdf conditional that works the same way.
% When switching from latex to pdflatex and vice-versa, the compiler may
% have to be run twice to clear warning/error messages.






% *** CITATION PACKAGES ***
%
% \ifCLASSOPTIONcompsoc
  % IEEE Computer Society needs nocompress option
  % requires cite.sty v4.0 or later (November 2003)
%   \usepackage[nocompress,noadjust]{cite}
% \else
  % normal IEEE
%  \usepackage[noadjust]{cite}
%\fi
% cite.sty was written by Donald Arseneau
% V1.6 and later of IEEEtran pre-defines the format of the cite.sty package
% \cite{} output to follow that of IEEE. Loading the cite package will
% result in citation numbers being automatically sorted and properly
% "compressed/ranged". e.g., [1], [9], [2], [7], [5], [6] without using
% cite.sty will become [1], [2], [5]--[7], [9] using cite.sty. cite.sty's
% \cite will automatically add leading space, if needed. Use cite.sty's
% noadjust option (cite.sty V3.8 and later) if you want to turn this off
% such as if a citation ever needs to be enclosed in parenthesis.
% cite.sty is already installed on most LaTeX systems. Be sure and use
% version 5.0 (2009-03-20) and later if using hyperref.sty.
% The latest version can be obtained at:
% http://www.ctan.org/tex-archive/macros/latex/contrib/cite/
% The documentation is contained in the cite.sty file itself.
%
% Note that some packages require special options to format as the Computer
% Society requires. In particular, Computer Society  papers do not use
% compressed citation ranges as is done in typical IEEE papers
% (e.g., [1]-[4]). Instead, they list every citation separately in order
% (e.g., [1], [2], [3], [4]). To get the latter we need to load the cite
% package with the nocompress option which is supported by cite.sty v4.0
% and later. Note also the use of a CLASSOPTION conditional provided by
% IEEEtran.cls V1.7 and later.





% *** GRAPHICS RELATED PACKAGES ***
%
%\ifCLASSINFOpdf
  % \usepackage[pdftex]{graphicx}
  % declare the path(s) where your graphic files are
  % \graphicspath{{../pdf/}{../jpeg/}}
  % and their extensions so you won't have to specify these with
  % every instance of \includegraphics
  % \DeclareGraphicsExtensions{.pdf,.jpeg,.png}
%\else
  % or other class option (dvipsone, dvipdf, if not using dvips). graphicx
  % will default to the driver specified in the system graphics.cfg if no
  % driver is specified.
  % \usepackage[dvips]{graphicx}
  % declare the path(s) where your graphic files are
  % \graphicspath{{../eps/}}
  % and their extensions so you won't have to specify these with
  % every instance of \includegraphics
  % \DeclareGraphicsExtensions{.eps}
%\fi
% graphicx was written by David Carlisle and Sebastian Rahtz. It is
% required if you want graphics, photos, etc. graphicx.sty is already
% installed on most LaTeX systems. The latest version and documentation
% can be obtained at: 
% http://www.ctan.org/tex-archive/macros/latex/required/graphics/
% Another good source of documentation is "Using Imported Graphics in
% LaTeX2e" by Keith Reckdahl which can be found at:
% http://www.ctan.org/tex-archive/info/epslatex/
%
% latex, and pdflatex in dvi mode, support graphics in encapsulated
% postscript (.eps) format. pdflatex in pdf mode supports graphics
% in .pdf, .jpeg, .png and .mps (metapost) formats. Users should ensure
% that all non-photo figures use a vector format (.eps, .pdf, .mps) and
% not a bitmapped formats (.jpeg, .png). IEEE frowns on bitmapped formats
% which can result in "jaggedy"/blurry rendering of lines and letters as
% well as large increases in file sizes.
%
% You can find documentation about the pdfTeX application at:
% http://www.tug.org/applications/pdftex






% *** MATH PACKAGES ***
%
\usepackage[cmex10]{amsmath}
% A popular package from the American Mathematical Society that provides
% many useful and powerful commands for dealing with mathematics. If using
% it, be sure to load this package with the cmex10 option to ensure that
% only type 1 fonts will utilized at all point sizes. Without this option,
% it is possible that some math symbols, particularly those within
% footnotes, will be rendered in bitmap form which will result in a
% document that can not be IEEE Xplore compliant!
%
% Also, note that the amsmath package sets \interdisplaylinepenalty to 10000
% thus preventing page breaks from occurring within multiline equations. Use:
%\interdisplaylinepenalty=2500
% after loading amsmath to restore such page breaks as IEEEtran.cls normally
% does. amsmath.sty is already installed on most LaTeX systems. The latest
% version and documentation can be obtained at:
% http://www.ctan.org/tex-archive/macros/latex/required/amslatex/math/





% *** SPECIALIZED LIST PACKAGES ***
%
%\usepackage{algorithmic}
% algorithmic.sty was written by Peter Williams and Rogerio Brito.
% This package provides an algorithmic environment fo describing algorithms.
% You can use the algorithmic environment in-text or within a figure
% environment to provide for a floating algorithm. Do NOT use the algorithm
% floating environment provided by algorithm.sty (by the same authors) or
% algorithm2e.sty (by Christophe Fiorio) as IEEE does not use dedicated
% algorithm float types and packages that provide these will not provide
% correct IEEE style captions. The latest version and documentation of
% algorithmic.sty can be obtained at:
% http://www.ctan.org/tex-archive/macros/latex/contrib/algorithms/
% There is also a support site at:
% http://algorithms.berlios.de/index.html
% Also of interest may be the (relatively newer and more customizable)
% algorithmicx.sty package by Szasz Janos:
% http://www.ctan.org/tex-archive/macros/latex/contrib/algorithmicx/

\usepackage{algpseudocode}


% *** ALIGNMENT PACKAGES ***
%
%\usepackage{array}
% Frank Mittelbach's and David Carlisle's array.sty patches and improves
% the standard LaTeX2e array and tabular environments to provide better
% appearance and additional user controls. As the default LaTeX2e table
% generation code is lacking to the point of almost being broken with
% respect to the quality of the end results, all users are strongly
% advised to use an enhanced (at the very least that provided by array.sty)
% set of table tools. array.sty is already installed on most systems. The
% latest version and documentation can be obtained at:
% http://www.ctan.org/tex-archive/macros/latex/required/tools/


% IEEEtran contains the IEEEeqnarray family of commands that can be used to
% generate multiline equations as well as matrices, tables, etc., of high
% quality.




% *** SUBFIGURE PACKAGES ***
%% \ifCLASSOPTIONcompsoc
%%   \usepackage[caption=false,font=footnotesize,labelfont=sf,textfont=sf]{subfig}
%% \else
%%   \usepackage[caption=false,font=footnotesize]{subfig}
%% \fi
% subfig.sty, written by Steven Douglas Cochran, is the modern replacement
% for subfigure.sty, the latter of which is no longer maintained and is
% incompatible with some LaTeX packages including fixltx2e. However,
% subfig.sty requires and automatically loads Axel Sommerfeldt's caption.sty
% which will override IEEEtran.cls' handling of captions and this will result
% in non-IEEE style figure/table captions. To prevent this problem, be sure
% and invoke subfig.sty's "caption=false" package option (available since
% subfig.sty version 1.3, 2005/06/28) as this is will preserve IEEEtran.cls
% handling of captions.
% Note that the Computer Society format requires a sans serif font rather
% than the serif font used in traditional IEEE formatting and thus the need
% to invoke different subfig.sty package options depending on whether
% compsoc mode has been enabled.
%
% The latest version and documentation of subfig.sty can be obtained at:
% http://www.ctan.org/tex-archive/macros/latex/contrib/subfig/




% *** FLOAT PACKAGES ***
%
%\usepackage{fixltx2e}
% fixltx2e, the successor to the earlier fix2col.sty, was written by
% Frank Mittelbach and David Carlisle. This package corrects a few problems
% in the LaTeX2e kernel, the most notable of which is that in current
% LaTeX2e releases, the ordering of single and double column floats is not
% guaranteed to be preserved. Thus, an unpatched LaTeX2e can allow a
% single column figure to be placed prior to an earlier double column
% figure. The latest version and documentation can be found at:
% http://www.ctan.org/tex-archive/macros/latex/base/


%\usepackage{stfloats}
% stfloats.sty was written by Sigitas Tolusis. This package gives LaTeX2e
% the ability to do double column floats at the bottom of the page as well
% as the top. (e.g., "\begin{figure*}[!b]" is not normally possible in
% LaTeX2e). It also provides a command:
%\fnbelowfloat
% to enable the placement of footnotes below bottom floats (the standard
% LaTeX2e kernel puts them above bottom floats). This is an invasive package
% which rewrites many portions of the LaTeX2e float routines. It may not work
% with other packages that modify the LaTeX2e float routines. The latest
% version and documentation can be obtained at:
% http://www.ctan.org/tex-archive/macros/latex/contrib/sttools/
% Do not use the stfloats baselinefloat ability as IEEE does not allow
% \baselineskip to stretch. Authors submitting work to the IEEE should note
% that IEEE rarely uses double column equations and that authors should try
% to avoid such use. Do not be tempted to use the cuted.sty or midfloat.sty
% packages (also by Sigitas Tolusis) as IEEE does not format its papers in
% such ways.
% Do not attempt to use stfloats with fixltx2e as they are incompatible.
% Instead, use Morten Hogholm'a dblfloatfix which combines the features
% of both fixltx2e and stfloats:
%
\usepackage{dblfloatfix}
% The latest version can be found at:
% http://www.ctan.org/tex-archive/macros/latex/contrib/dblfloatfix/




%\ifCLASSOPTIONcaptionsoff
%  \usepackage[nomarkers]{endfloat}
% \let\MYoriglatexcaption\caption
% \renewcommand{\caption}[2][\relax]{\MYoriglatexcaption[#2]{#2}}
%\fi
% endfloat.sty was written by James Darrell McCauley, Jeff Goldberg and 
% Axel Sommerfeldt. This package may be useful when used in conjunction with 
% IEEEtran.cls'  captionsoff option. Some IEEE journals/societies require that
% submissions have lists of figures/tables at the end of the paper and that
% figures/tables without any captions are placed on a page by themselves at
% the end of the document. If needed, the draftcls IEEEtran class option or
% \CLASSINPUTbaselinestretch interface can be used to increase the line
% spacing as well. Be sure and use the nomarkers option of endfloat to
% prevent endfloat from "marking" where the figures would have been placed
% in the text. The two hack lines of code above are a slight modification of
% that suggested by in the endfloat docs (section 8.4.1) to ensure that
% the full captions always appear in the list of figures/tables - even if
% the user used the short optional argument of \caption[]{}.
% IEEE papers do not typically make use of \caption[]'s optional argument,
% so this should not be an issue. A similar trick can be used to disable
% captions of packages such as subfig.sty that lack options to turn off
% the subcaptions:
% For subfig.sty:
% \let\MYorigsubfloat\subfloat
% \renewcommand{\subfloat}[2][\relax]{\MYorigsubfloat[]{#2}}
% However, the above trick will not work if both optional arguments of
% the \subfloat command are used. Furthermore, there needs to be a
% description of each subfigure *somewhere* and endfloat does not add
% subfigure captions to its list of figures. Thus, the best approach is to
% avoid the use of subfigure captions (many IEEE journals avoid them anyway)
% and instead reference/explain all the subfigures within the main caption.
% The latest version of endfloat.sty and its documentation can obtained at:
% http://www.ctan.org/tex-archive/macros/latex/contrib/endfloat/
%
% The IEEEtran \ifCLASSOPTIONcaptionsoff conditional can also be used
% later in the document, say, to conditionally put the References on a 
% page by themselves.




% *** PDF, URL AND HYPERLINK PACKAGES ***
%
%\usepackage{url}
% url.sty was written by Donald Arseneau. It provides better support for
% handling and breaking URLs. url.sty is already installed on most LaTeX
% systems. The latest version and documentation can be obtained at:
% http://www.ctan.org/tex-archive/macros/latex/contrib/url/
% Basically, \url{my_url_here}.





% *** Do not adjust lengths that control margins, column widths, etc. ***
% *** Do not use packages that alter fonts (such as pslatex).         ***
% There should be no need to do such things with IEEEtran.cls V1.6 and later.
% (Unless specifically asked to do so by the journal or conference you plan
% to submit to, of course. )

\usepackage{alltt}
\usepackage{tikz}
\usetikzlibrary{arrows.meta}
\usepackage{multicol}
\newcommand{\hide}[1]{\ignorespaces}
\newcommand{\jx}[1]{{\textbf{Jiaxiang: }}#1{ \textbf{ End}}}
\newtheorem{theorem}{Theorem}
\newtheorem{lemma}{Lemma}
\newcommand{\ANSWER}{\medskip\noindent\textbf{RESPONSE: }}
\newcommand{\COMMENT}{\medskip\noindent\textbf{COMMENT: }}

% Define the fontsize in environment {verbatim}
%% \makeatletter
%% \def\verbatim{\small\@verbatim \frenchspacing\@vobeyspaces \@xverbatim}
%% %\def\verbatim@font{\small\ttfamily}
%% \makeatother


% correct bad hyphenation here
\hyphenation{op-tical net-works semi-conduc-tor}

\begin{document}

\title{Responses to Reviews}
\date{}

\maketitle
\vspace{-30pt}
\begin{center}
  \begin{tabular}{r l}
    \hline\hline
    & \\
    Manuscript Number: & 15-TIE-3480.R1\\
    Manuscript Title: & Formal Modeling and Verification of a \\
    & Rate-Monotonic Scheduling Implementation \\
    & with Real-Time Maude \\
    Submitted to: & Transactions on Industrial Electronics \\
    Manuscript Type: & Regular paper \\
    & \\
    \hline\hline
  \end{tabular}
  \bigskip
\end{center}

\newcommand{\by}{$\rightarrow$ }

\begin{itemize}\setlength{\itemsep}{10pt}

\item Page 1, line 3: ``With'' \by ``with''.

We guess the first letter of a preposition need not be capitalized.

\item Page 1, line 65: ``few attempt [23], [24]'' \by ``few [23], [24] attempt''

\item Page 1, line 68: ``\emph{real-time Maude}'' \by ``\emph{Real-Time Maude}''.

It is the name of the tool.

\item Page 1, line 76: ``Section'' and ``II''. Please do not break
  ``Section II'' into two lines.

\item Page 2, line 78: ``real-time Maude'' \by ``Real-Time Maude''

\item Page 2, line 80: ``real-time Maude'' \by ``Real-Time Maude''
  
\item Page 2, line 110: ``Real-time Maude'' \by ``Real-Time Maude''

\item Page 2, line 113: ``Real-time Maude'' \by ``Real-Time Maude''

\item Page 2, line 127: ``$IR$'' \by ``$\mathit{IR}$''. Please use the
  macro \verb|\mathit{}|.

\item Page 2, line 134: ``$TR$'' \by ``$\mathit{TR}$''. Please use the
  macro \verb|\mathit{}|.

\item Page 2, line 138: ``$IR$'' \by ``$\mathit{IR}$''. Please use the
  macro \verb|\mathit{}|.  

\item Page 2, line 138: ``$TR$'' \by ``$\mathit{TR}$''. Please use the
  macro \verb|\mathit{}|.

\item Page 2, line 139: ``nondeterministically'' \by
  ``non-deterministically''.

\item Page 2, line 141: ``nondeterministic'' \by
  ``non-deterministic''.

\item Page 2, line 142: ``real-time Maude'' \by ``Real-Time Maude''

\item Page 2, line 143: ``Real-time Maude'' \by ``Real-Time Maude''

\item Page 2, line 144: ``\verb|class ...|'' \by ``$\texttt{class
}C\texttt{ |
}att_1\texttt{:}s_1\texttt{,}\ldots\texttt{,}att_n\texttt{:}s_n$''.

All things should be put in \verb|\verb| mode except ``$C$''
``$att_1$'' ``$s_1$'' ``$\ldots$'' ``$att_n$'' and ``$s_n$'', which
should be in math mode.

\item Page 2, line 147: ``\verb|< ... >|'' \by ``$\texttt{< }
  O\texttt{:} C \texttt{ | } att_1\texttt{:}val_1\texttt{,} \ldots
  \texttt{,}att_n\texttt{:}val_n\texttt{ >}$''. 

All things should be put in \verb|\verb| mode except ``$O$'' ``$C$''
``$att_1$'' ``$val_1$'' ``$\ldots$'' ``$att_n$'' and ``$val_n$'',
which should be in math mode.

\item Page 2, line 152: ``Real-time Maude'' \by ``Real-Time Maude''

\item Page 2, line 159: ``real-time Maude'' \by ``Real-Time Maude''

\item Page 2, line 164: ``\verb|ceq ...|'' \by ``$\texttt{ceq }
  \mathit{statePattern} \texttt{ |= } \mathit{prop} \texttt{ = } b
  \texttt{ if } cond$''.

All things should be put in \verb|\verb| mode except
``$\mathit{statePattern}$'' ``$\mathit{prop}$'' ``$b$'' and
``$cond$'', which should be in math mode. And ``$\mathit{statePattern}$''
and ``$\mathit{prop}$'' are better to use the macro \verb|\mathit{}|.

\item Page 2, line 165: ``$prop$'' \by ``$\mathit{prop}$''. Please use the
  macro \verb|\mathit{}|.

\item Page 2, line 166: ``$statePattern$'' \by
  ``$\mathit{statePattern}$''. Please use the macro \verb|\mathit{}|.

\item Page 2, line 167: ``$prop$'' \by ``$\mathit{prop}$''. Please use the
  macro \verb|\mathit{}|.

\item Page 2, line 168: ``$s~|= prop$'' \by ``$s \texttt{ |= }
  \mathit{prop}$''.

All things should be put in \verb|\verb| mode except ``$s$'' and
``$\mathit{prop}$'', which should be in math mode. And please use the
macro \verb|\mathit{}| for ``$\mathit{prop}$''.

\item Page 2, line 170: ``(negation)'' \by ``\verb|~|(negation)''. The
  tilde ``\verb|~|'' should be in \verb|\verb| mode.

\item Page 2, line 170: ``$\backslash$\verb|/|'' \by ``\verb|\/|''. All
  things should be put in \verb|\verb| mode.

\item Page 2, line 171: ``Real-time Maude'' \by ``Real-Time Maude''

\item Page 2, line between 172 and 173: ``(mc ... )'' \by
  ``\verb|(mc |$s$\verb+ |=u +$\Phi$\verb| .)|''.

All things should be put in \verb|\verb| mode except ``$s$'' and
``$\Phi$'', which should be in math mode.

\item Page 3, line 210--217:

For the \verb|\enumerate| environment, is it possible to reduce the
indentation right after the number, such that for each item, the first
word of the first line is aligned with those of the other lines? For
example, reduce the space between ``1)'' and ``the'' at line 210, such
that this ``the'' is aligned with ``possibly'' at line 211 and ``is''
at line 212.

\item Page 3, line 218: ``Switching''.

Please remove the indentation before ``Switching'', this is not a new
paragraph.

\item Page 3, line 222--227:

For the \verb|\enumerate| environment, is it possible to reduce the
indentation right after the number, such that for each item, the first
word of the first line is aligned with those of the other lines? 

\item Page 3, line 258: ``Sections IV-C''. Please do not break
  ``Sections IV-C'' into two lines.

\item Page 3, line 259: ``Section IV-F''. Please do not break
  ``Section IV-F'' into two lines.

\item Page 4, lines 293--294:

Please make sure that this kind of code throughout the article is put
in \verb|\verb| mode. (We are not sure whether it is true because
``\verb+|+'' look a bit higher and ``\verb|>|'' look a bit wider than
what they should be.)

\item Page 4, lines 300--302: it should be
\begin{verbatim}
  class PTask | priority : Nat, 
                period : Nat, 
                status : Status .
\end{verbatim}

Please make sure that ``\verb|priority|'' ``\verb|period|'' and
``\verb|status|'' are aligned with each other.

\item Page 4, line 304: ``\verb|RUN-NING|'' \by ``\verb|RUNNING|''.

Please use \verb|\verb| mode and do not break the word into two parts.

\item Page 4, lines 320--321: it should be
\begin{verbatim}
  class Regs | pc : TaskID, 
               mask : Bool, ir : Bool .
\end{verbatim}

Please make sure that ``\verb|pc|'' is aligned with ``\verb|mask|''.

\item Page 4, lines 331--332: it should be
\begin{verbatim}
  class IntSrc | val : Time, 
                 cycle : Time .
\end{verbatim}

Please make sure that ``\verb|val|'' is aligned with ``\verb|cycle|''.

\item Page 4, lines 335--337: it should be
\begin{verbatim}
  op _____ : TaskList Timer SysTasks 
             Hardware Object 
             ~> System [ctor] .
\end{verbatim}

Please make sure that ``\verb|TaskList|'' ``\verb|Hardware|'' and
``\verb|~>|'' are aligned with each other.


\item Page 4, lines 338--339: it should be
\begin{verbatim}
  mb (L T STS HW < O : IntSrc |>) 
     : System .
\end{verbatim}

Please make sure that ``\verb|(|'' is aligned with ``\verb|:|''.

\item Page 4, line 340: ``\verb| ~>|'' \by ``\verb|~>|''. Please
  remove the space before ``\verb|~|''.

\item Page 4, line 342: ``\verb|Sys-Tasks|'' \by ``\verb|SysTasks|''.

Please use \verb|\verb| mode and do not break the word into two parts.


\item Page 5, lines 362--363: it should be
\begin{verbatim}
  op updateStatus_with_ : TaskList Timer 
                            -> TaskList . 
\end{verbatim}

Please make sure that ``\verb|->|'' is aligned with the first
``\verb|s|'' in ``\verb|TaskList|'' at line 362.

\item Page 5, lines 366--367: it should be
\begin{verbatim}
  op update_with_ : Object Timer 
                      ~> Object .
\end{verbatim}

Please make sure that ``\verb|~>|'' is aligned with the ``\verb|j|''
in ``\verb|Object|'' at line 366.


\item Page 5, lines 371--374: it should be
\begin{verbatim}
      = if ST == DORMANT 
        then < O : PTask | status : READY >
        else error fi
      if TIMER rem T == 0 .
\end{verbatim}

Please make sure that ``\verb|if|'' at 371, ``\verb|then|'' at 372 and
``\verb|else|'' at 373 are aligned with each other. And make sure that
``\verb|=|'' at 371 is aligned with ``\verb|if|'' at 374.


\item Page 5, lines 377--381: it should be
\begin{verbatim}
     = if ST == RUNNING 
       then < O : PTask 
              | status : INTERRUPT >
       else < O : PTask |> fi 
     [otherwise] .
\end{verbatim}

Please make sure that ``\verb|if|'' at 377, ``\verb|then|'' at 378 and
``\verb|else|'' at 380 are aligned with each other. Make sure that
``\verb+|+'' at 379 is aligned with ``\verb|O|'' at 378. Make sure
that ``\verb|[otherwise]|'' at 381 is aligned with ``\verb|=|'' at
377.


\item Page 5, lines 400--403: it should be
\begin{verbatim}
  crl [interrupt-handle] :
    SYSTEM =>  
    ((SYSTEM).interrupt).startScheduling
    if (SYSTEM).existInt .
\end{verbatim}

Please make sure that ``\verb|SYSTEM|'' at 401, ``\verb|((|'' at 402
and ``\verb|if|'' at 403 are aligned with each other.


\item Page 5, line 407--413:

For the \verb|\enumerate| environment, is it possible to reduce the
indentation right after the number, such that for each item, the first
word of the first line is aligned with those of the other lines? 


\item Page 5, lines 426--427: it should be
\begin{verbatim}
  op _.startScheduling : System 
                           -> System .
\end{verbatim}

Please make sure that ``\verb|->|'' at 427 is aligned with the second
``\verb|s|'' in ``\verb|System|'' at 426.


\item Page 5, lines 429--430: it should be
\begin{verbatim}
     = ((updateStatus L with T) 
        inc(T) STS HW ISRC) .
\end{verbatim}

Please make sure that ``\verb|inc|'' at 430 is aligned with the second
``\verb|(|'' at 429.


\item Page 5, lines 435--436: it should be
\begin{verbatim}
  op _.finishScheduling : System 
                            -> System .
\end{verbatim}

Please make sure that ``\verb|->|'' at 436 is aligned with the second
``\verb|s|'' in ``\verb|System|'' at 435.


\item Page 5, lines 438--439: it should be
\begin{verbatim}
     = (L T (finish scheduling in STS) 
        HW ISRC).run1stTask .
\end{verbatim}

Please make sure that ``\verb|HW|'' at 439 is aligned with
``\verb|L|'' at 438.


\item Page 5, line 450: ``\verb|/|$\backslash$'' \by
  ``\verb|/\|''. All things should be put in \verb|\verb| mode.

\item Page 5, line 451: ``\verb|/|$\backslash$'' \by
  ``\verb|/\|''. All things should be put in \verb|\verb| mode.

\item Page 6, footnote 3: ``real-time Maude'' \by ``Real-Time Maude''

\item Page 6, footnote 4: ``Real-time Maude'' \by ``Real-Time Maude''  

\item Page 6, line 478: ``$ID$'' \by ``$\mathit{ID}$''. Please use the
  macro \verb|\mathit{}|.


\item Page 6, lines 482--485: it should be
\begin{verbatim}
      = (deltaTask(ID, L, R) 
         T STS HW (deltaIS(ISRC, R)))
      if ID := (HW).getPc 
         /\ ID :: MaybeNat .
\end{verbatim}

Please make sure that ``\verb|deltaTask|'' at 482 is aligned with
``\verb|T|'' at 483. Make sure that ``\verb|=|'' at 482 is aligned
with ``\verb|if|'' at 484. And make sure that ``\verb|/\|'' at 485 is
aligned with ``\verb|ID|'' at 484.


\item Page 6, lines 495--498: it should be
\begin{verbatim}
      = minimum(mteTask(ID, L),
                mteIS(ISRC), mteIr(HW))
      if ID := (HW).getPc 
         /\ ID :: MaybeNat .
\end{verbatim}

Please make sure that ``\verb|mteTask|'' at 495 is aligned with
``\verb|mteIS|'' at 496. Make sure that ``\verb|=|'' at 495 is aligned
with ``\verb|if|'' at 497. And make sure that ``\verb|/\|'' at 498 is
aligned with ``\verb|ID|'' at 497.

\item Page 6, line 524: ``\verb|[]( taskTimeout)|'' \by
  ``\verb|[](~taskTimeout)|''.

All things should be put in \verb|\verb| mode.

\item Page 6, line 530: ``\verb|( taskTimeout)|'' \by
  ``\verb|(~taskTimeout)|''.

All things should be put in \verb|\verb| mode.  


\item Page 6, lines 537--540: it should be
\begin{verbatim}
      = if ID :: MaybeNat 
        then shouldRun(ID, L)
        else true fi
      if ID := (HW).getPc .
\end{verbatim}

Please make sure that ``\verb|if|'' at 537, ``\verb|then|'' at 538 and
``\verb|else|'' at 539 are aligned with each other.  And make sure
that ``\verb|=|'' at 537 is aligned with ``\verb|if|'' at 540.


\item Page 6, lines 546--547: ``\verb|(...)\/(...)|'' \by
  ``\verb|([]correct)\/(correct U taskTimeout)|''.

Please put the formula in \verb|\verb| mode, and do not break it into
two lines.


\item Page 6, lines 551--552: it should be
\begin{verbatim}
  (mc init |=u ([]correct) 
               \/ (correct U taskTimeout) .)
\end{verbatim}

Please make sure that ``\verb|\/|'' at 552 is aligned with
``\verb|([]correct)|'' at 551.

Or if it is too wide to fit the column, it can be
\begin{verbatim}
  (mc init |=u ([]correct) \/ 
               (correct U taskTimeout) .)
\end{verbatim}

Please make sure that ``\verb|(|'' at 552 is aligned with the second
``\verb|(|'' at 551.


\item Page 6, lines 556--559:

If possible, we prefer an \verb|\itemize| (unnumbered) environment
instead of an \verb|\enumerate| (numbered) enviroment here.

\item Page 6, line 556: ``5 ms''. The space seems larger than others,
  please correct it.

\item Page 6, line 557: ``38 $\mu$s'' and ``20 $\mu$s''. The space
  seems larger than others, please correct it.

\item Page 7, lines 564--573:

If possible, we prefer an \verb|\itemize| (unnumbered) environment
instead of an \verb|\enumerate| (numbered) enviroment here.

\item Page 7, lines 564--573:

Along these lines, the spaces between the numbers and their units seem
larger than the normal value, please correct them.

\item Page 7, line 574: ``real-time Maude'' \by ``Real-Time Maude''

\item Page 7, footnote 5: ``time robustness'' \by ``time-robustness''

\item Page 7, footnote 5: ``tick invariance'' \by ``tick-invariance''
  
\item Page 7, line 635: ``time robust'' \by ``time-robust''

\item Page 7, line 637: ``tick invariant'' \by ``tick-invariant''

\item Page 7, line 665: ``nondeterminism'' \by ``non-determinism''

\item Page 8, line 672: ``nondeterminism'' \by ``non-determinism''

\item Page 8, line 697: ``such as'' \by ``similar to''

\item Page 8, line 700: ``real-time Maude'' \by ``Real-Time Maude''

\item Page 8, line 704: ``real-time Maude'' \by ``Real-Time Maude''

\item Page 8, line 706: ``real-time Maude'' \by ``Real-Time Maude''

\item Page 8, line 724: ``time robustness'' \by ``time-robustness''

\item Page 8, line 725: ``time robustness'' \by ``time-robustness''

\item Page 8, line 729: ``time robust'' \by ``time-robust''

\item Page 8, lines 731--738. Please use ``(i)'' ``(ii)'' instead of
  ``1)'' ``2)'' for the numbers, because they correspond to the
  \emph{Proof of Theorem 2}.

\item Page 8, line 731: ``\verb|mte(...)|$=$\verb|mte(...)|'' \by
  ``\verb|mte(delta(|$t$\verb|,|$r$\verb|))| $=$
  \verb|mte(|$t$\verb|)|''

\item Page 8, line 732: ``$r\le$\verb|mte(|$t$\verb|)|'' \by ``$r\le$
  \verb|mte(|$t$\verb|)|''

\item Page 8, line 734: ``\verb|delta(|$t$ \verb|,|$0$\verb|)|'' \by
  ``\verb|delta(|$t$\verb|,|$0$\verb|)|''

\item Page 8, line 735:

``\verb|delta(delta(|$t$ \verb|,|$r$ \verb|),|$r'$ \verb|)|
  $=$\verb|delta(|$t$\verb|,|$r+r'$\verb|)|''

\by
  
``\verb|delta(delta(|$t$\verb|,|$r$\verb|),|$r'$\verb|)| $=$
\verb|delta(|$t$\verb|,|$r+r'$\verb|)|''

\end{itemize}


\end{document}



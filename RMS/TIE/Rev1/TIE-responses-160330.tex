\documentclass[10pt,journal]{IEEEtran}
%
% If IEEEtran.cls has not been installed into the LaTeX system files,
% manually specify the path to it like:
% \documentclass[10pt,journal,compsoc]{../sty/IEEEtran}





% Some very useful LaTeX packages include:
% (uncomment the ones you want to load)


% *** MISC UTILITY PACKAGES ***
%
%\usepackage{ifpdf}
% Heiko Oberdiek's ifpdf.sty is very useful if you need conditional
% compilation based on whether the output is pdf or dvi.
% usage:
% \ifpdf
%   % pdf code
% \else
%   % dvi code
% \fi
% The latest version of ifpdf.sty can be obtained from:
% http://www.ctan.org/tex-archive/macros/latex/contrib/oberdiek/
% Also, note that IEEEtran.cls V1.7 and later provides a builtin
% \ifCLASSINFOpdf conditional that works the same way.
% When switching from latex to pdflatex and vice-versa, the compiler may
% have to be run twice to clear warning/error messages.






% *** CITATION PACKAGES ***
%
\ifCLASSOPTIONcompsoc
  % IEEE Computer Society needs nocompress option
  % requires cite.sty v4.0 or later (November 2003)
  \usepackage[nocompress,noadjust]{cite}
\else
  % normal IEEE
  \usepackage[noadjust]{cite}
\fi
% cite.sty was written by Donald Arseneau
% V1.6 and later of IEEEtran pre-defines the format of the cite.sty package
% \cite{} output to follow that of IEEE. Loading the cite package will
% result in citation numbers being automatically sorted and properly
% "compressed/ranged". e.g., [1], [9], [2], [7], [5], [6] without using
% cite.sty will become [1], [2], [5]--[7], [9] using cite.sty. cite.sty's
% \cite will automatically add leading space, if needed. Use cite.sty's
% noadjust option (cite.sty V3.8 and later) if you want to turn this off
% such as if a citation ever needs to be enclosed in parenthesis.
% cite.sty is already installed on most LaTeX systems. Be sure and use
% version 5.0 (2009-03-20) and later if using hyperref.sty.
% The latest version can be obtained at:
% http://www.ctan.org/tex-archive/macros/latex/contrib/cite/
% The documentation is contained in the cite.sty file itself.
%
% Note that some packages require special options to format as the Computer
% Society requires. In particular, Computer Society  papers do not use
% compressed citation ranges as is done in typical IEEE papers
% (e.g., [1]-[4]). Instead, they list every citation separately in order
% (e.g., [1], [2], [3], [4]). To get the latter we need to load the cite
% package with the nocompress option which is supported by cite.sty v4.0
% and later. Note also the use of a CLASSOPTION conditional provided by
% IEEEtran.cls V1.7 and later.





% *** GRAPHICS RELATED PACKAGES ***
%
\ifCLASSINFOpdf
  % \usepackage[pdftex]{graphicx}
  % declare the path(s) where your graphic files are
  % \graphicspath{{../pdf/}{../jpeg/}}
  % and their extensions so you won't have to specify these with
  % every instance of \includegraphics
  % \DeclareGraphicsExtensions{.pdf,.jpeg,.png}
\else
  % or other class option (dvipsone, dvipdf, if not using dvips). graphicx
  % will default to the driver specified in the system graphics.cfg if no
  % driver is specified.
  % \usepackage[dvips]{graphicx}
  % declare the path(s) where your graphic files are
  % \graphicspath{{../eps/}}
  % and their extensions so you won't have to specify these with
  % every instance of \includegraphics
  % \DeclareGraphicsExtensions{.eps}
\fi
% graphicx was written by David Carlisle and Sebastian Rahtz. It is
% required if you want graphics, photos, etc. graphicx.sty is already
% installed on most LaTeX systems. The latest version and documentation
% can be obtained at: 
% http://www.ctan.org/tex-archive/macros/latex/required/graphics/
% Another good source of documentation is "Using Imported Graphics in
% LaTeX2e" by Keith Reckdahl which can be found at:
% http://www.ctan.org/tex-archive/info/epslatex/
%
% latex, and pdflatex in dvi mode, support graphics in encapsulated
% postscript (.eps) format. pdflatex in pdf mode supports graphics
% in .pdf, .jpeg, .png and .mps (metapost) formats. Users should ensure
% that all non-photo figures use a vector format (.eps, .pdf, .mps) and
% not a bitmapped formats (.jpeg, .png). IEEE frowns on bitmapped formats
% which can result in "jaggedy"/blurry rendering of lines and letters as
% well as large increases in file sizes.
%
% You can find documentation about the pdfTeX application at:
% http://www.tug.org/applications/pdftex






% *** MATH PACKAGES ***
%
\usepackage[cmex10]{amsmath}
% A popular package from the American Mathematical Society that provides
% many useful and powerful commands for dealing with mathematics. If using
% it, be sure to load this package with the cmex10 option to ensure that
% only type 1 fonts will utilized at all point sizes. Without this option,
% it is possible that some math symbols, particularly those within
% footnotes, will be rendered in bitmap form which will result in a
% document that can not be IEEE Xplore compliant!
%
% Also, note that the amsmath package sets \interdisplaylinepenalty to 10000
% thus preventing page breaks from occurring within multiline equations. Use:
%\interdisplaylinepenalty=2500
% after loading amsmath to restore such page breaks as IEEEtran.cls normally
% does. amsmath.sty is already installed on most LaTeX systems. The latest
% version and documentation can be obtained at:
% http://www.ctan.org/tex-archive/macros/latex/required/amslatex/math/





% *** SPECIALIZED LIST PACKAGES ***
%
%\usepackage{algorithmic}
% algorithmic.sty was written by Peter Williams and Rogerio Brito.
% This package provides an algorithmic environment fo describing algorithms.
% You can use the algorithmic environment in-text or within a figure
% environment to provide for a floating algorithm. Do NOT use the algorithm
% floating environment provided by algorithm.sty (by the same authors) or
% algorithm2e.sty (by Christophe Fiorio) as IEEE does not use dedicated
% algorithm float types and packages that provide these will not provide
% correct IEEE style captions. The latest version and documentation of
% algorithmic.sty can be obtained at:
% http://www.ctan.org/tex-archive/macros/latex/contrib/algorithms/
% There is also a support site at:
% http://algorithms.berlios.de/index.html
% Also of interest may be the (relatively newer and more customizable)
% algorithmicx.sty package by Szasz Janos:
% http://www.ctan.org/tex-archive/macros/latex/contrib/algorithmicx/

\usepackage{algpseudocode}


% *** ALIGNMENT PACKAGES ***
%
%\usepackage{array}
% Frank Mittelbach's and David Carlisle's array.sty patches and improves
% the standard LaTeX2e array and tabular environments to provide better
% appearance and additional user controls. As the default LaTeX2e table
% generation code is lacking to the point of almost being broken with
% respect to the quality of the end results, all users are strongly
% advised to use an enhanced (at the very least that provided by array.sty)
% set of table tools. array.sty is already installed on most systems. The
% latest version and documentation can be obtained at:
% http://www.ctan.org/tex-archive/macros/latex/required/tools/


% IEEEtran contains the IEEEeqnarray family of commands that can be used to
% generate multiline equations as well as matrices, tables, etc., of high
% quality.




% *** SUBFIGURE PACKAGES ***
\ifCLASSOPTIONcompsoc
  \usepackage[caption=false,font=footnotesize,labelfont=sf,textfont=sf]{subfig}
\else
  \usepackage[caption=false,font=footnotesize]{subfig}
\fi
% subfig.sty, written by Steven Douglas Cochran, is the modern replacement
% for subfigure.sty, the latter of which is no longer maintained and is
% incompatible with some LaTeX packages including fixltx2e. However,
% subfig.sty requires and automatically loads Axel Sommerfeldt's caption.sty
% which will override IEEEtran.cls' handling of captions and this will result
% in non-IEEE style figure/table captions. To prevent this problem, be sure
% and invoke subfig.sty's "caption=false" package option (available since
% subfig.sty version 1.3, 2005/06/28) as this is will preserve IEEEtran.cls
% handling of captions.
% Note that the Computer Society format requires a sans serif font rather
% than the serif font used in traditional IEEE formatting and thus the need
% to invoke different subfig.sty package options depending on whether
% compsoc mode has been enabled.
%
% The latest version and documentation of subfig.sty can be obtained at:
% http://www.ctan.org/tex-archive/macros/latex/contrib/subfig/




% *** FLOAT PACKAGES ***
%
%\usepackage{fixltx2e}
% fixltx2e, the successor to the earlier fix2col.sty, was written by
% Frank Mittelbach and David Carlisle. This package corrects a few problems
% in the LaTeX2e kernel, the most notable of which is that in current
% LaTeX2e releases, the ordering of single and double column floats is not
% guaranteed to be preserved. Thus, an unpatched LaTeX2e can allow a
% single column figure to be placed prior to an earlier double column
% figure. The latest version and documentation can be found at:
% http://www.ctan.org/tex-archive/macros/latex/base/


%\usepackage{stfloats}
% stfloats.sty was written by Sigitas Tolusis. This package gives LaTeX2e
% the ability to do double column floats at the bottom of the page as well
% as the top. (e.g., "\begin{figure*}[!b]" is not normally possible in
% LaTeX2e). It also provides a command:
%\fnbelowfloat
% to enable the placement of footnotes below bottom floats (the standard
% LaTeX2e kernel puts them above bottom floats). This is an invasive package
% which rewrites many portions of the LaTeX2e float routines. It may not work
% with other packages that modify the LaTeX2e float routines. The latest
% version and documentation can be obtained at:
% http://www.ctan.org/tex-archive/macros/latex/contrib/sttools/
% Do not use the stfloats baselinefloat ability as IEEE does not allow
% \baselineskip to stretch. Authors submitting work to the IEEE should note
% that IEEE rarely uses double column equations and that authors should try
% to avoid such use. Do not be tempted to use the cuted.sty or midfloat.sty
% packages (also by Sigitas Tolusis) as IEEE does not format its papers in
% such ways.
% Do not attempt to use stfloats with fixltx2e as they are incompatible.
% Instead, use Morten Hogholm'a dblfloatfix which combines the features
% of both fixltx2e and stfloats:
%
\usepackage{dblfloatfix}
% The latest version can be found at:
% http://www.ctan.org/tex-archive/macros/latex/contrib/dblfloatfix/




%\ifCLASSOPTIONcaptionsoff
%  \usepackage[nomarkers]{endfloat}
% \let\MYoriglatexcaption\caption
% \renewcommand{\caption}[2][\relax]{\MYoriglatexcaption[#2]{#2}}
%\fi
% endfloat.sty was written by James Darrell McCauley, Jeff Goldberg and 
% Axel Sommerfeldt. This package may be useful when used in conjunction with 
% IEEEtran.cls'  captionsoff option. Some IEEE journals/societies require that
% submissions have lists of figures/tables at the end of the paper and that
% figures/tables without any captions are placed on a page by themselves at
% the end of the document. If needed, the draftcls IEEEtran class option or
% \CLASSINPUTbaselinestretch interface can be used to increase the line
% spacing as well. Be sure and use the nomarkers option of endfloat to
% prevent endfloat from "marking" where the figures would have been placed
% in the text. The two hack lines of code above are a slight modification of
% that suggested by in the endfloat docs (section 8.4.1) to ensure that
% the full captions always appear in the list of figures/tables - even if
% the user used the short optional argument of \caption[]{}.
% IEEE papers do not typically make use of \caption[]'s optional argument,
% so this should not be an issue. A similar trick can be used to disable
% captions of packages such as subfig.sty that lack options to turn off
% the subcaptions:
% For subfig.sty:
% \let\MYorigsubfloat\subfloat
% \renewcommand{\subfloat}[2][\relax]{\MYorigsubfloat[]{#2}}
% However, the above trick will not work if both optional arguments of
% the \subfloat command are used. Furthermore, there needs to be a
% description of each subfigure *somewhere* and endfloat does not add
% subfigure captions to its list of figures. Thus, the best approach is to
% avoid the use of subfigure captions (many IEEE journals avoid them anyway)
% and instead reference/explain all the subfigures within the main caption.
% The latest version of endfloat.sty and its documentation can obtained at:
% http://www.ctan.org/tex-archive/macros/latex/contrib/endfloat/
%
% The IEEEtran \ifCLASSOPTIONcaptionsoff conditional can also be used
% later in the document, say, to conditionally put the References on a 
% page by themselves.




% *** PDF, URL AND HYPERLINK PACKAGES ***
%
%\usepackage{url}
% url.sty was written by Donald Arseneau. It provides better support for
% handling and breaking URLs. url.sty is already installed on most LaTeX
% systems. The latest version and documentation can be obtained at:
% http://www.ctan.org/tex-archive/macros/latex/contrib/url/
% Basically, \url{my_url_here}.





% *** Do not adjust lengths that control margins, column widths, etc. ***
% *** Do not use packages that alter fonts (such as pslatex).         ***
% There should be no need to do such things with IEEEtran.cls V1.6 and later.
% (Unless specifically asked to do so by the journal or conference you plan
% to submit to, of course. )

\usepackage{alltt}
\usepackage{tikz}
\usetikzlibrary{arrows.meta}
\usepackage{multicol}
\newcommand{\hide}[1]{\ignorespaces}
\newcommand{\jx}[1]{{\bf Jiaxiang: }#1{ \bf End}}
\newtheorem{theorem}{Theorem}
\newtheorem{lemma}{Lemma}
\newcommand{\ANSWER}{\medskip\noindent{\bf RESPONSE: }}
\newcommand{\COMMENT}{\medskip\noindent{\bf COMMENT: }}

% Define the fontsize in environment {verbatim}
\makeatletter
\def\verbatim{\small\@verbatim \frenchspacing\@vobeyspaces \@xverbatim}
%\def\verbatim@font{\small\ttfamily}
\makeatother


% correct bad hyphenation here
\hyphenation{op-tical net-works semi-conduc-tor}


\begin{document}
%
% paper title
% Titles are generally capitalized except for words such as a, an, and, as,
% at, but, by, for, in, nor, of, on, or, the, to and up, which are usually
% not capitalized unless they are the first or last word of the title.
% Linebreaks \\ can be used within to get better formatting as desired.
% Do not put math or special symbols in the title.
%\title{Corrections to Manuscript No. 15-TIE-3480}
\title{Responses to Reviews}
%
%
% author names and IEEE memberships
% note positions of commas and nonbreaking spaces ( ~ ) LaTeX will not break
% a structure at a ~ so this keeps an author's name from being broken across
% two lines.
% use \thanks{} to gain access to the first footnote area
% a separate \thanks must be used for each paragraph as LaTeX2e's \thanks
% was not built to handle multiple paragraphs
%
%
%\IEEEcompsocitemizethanks is a special \thanks that produces the bulleted
% lists the Computer Society journals use for "first footnote" author
% affiliations. Use \IEEEcompsocthanksitem which works much like \item
% for each affiliation group. When not in compsoc mode,
% \IEEEcompsocitemizethanks becomes like \thanks and
% \IEEEcompsocthanksitem becomes a line break with idention. This
% facilitates dual compilation, although admittedly the differences in the
% desired content of \author between the different types of papers makes a
% one-size-fits-all approach a daunting prospect. For instance, compsoc 
% journal papers have the author affiliations above the "Manuscript
% received ..."  text while in non-compsoc journals this is reversed. Sigh.

\hide{
\author{Jiaxiang~Liu,
        Min~Zhou,
        Xiaoyu~Song,
        Ming~Gu,
        and~Jiaguang~Sun% <-this % stops a space
\IEEEcompsocitemizethanks{%
\IEEEcompsocthanksitem J. Liu is with the School of Software, Tsinghua
University, Beijing 100084, China, and also with LIX, \'Ecole
Polytechnique, Palaiseau 91120, France.\protect\\ E-mail:
jiaxiang.liu@hotmail.com
\IEEEcompsocthanksitem M. Zhou, M. Gu and J. Sun are with the School
of Software, Tsinghua University, Beijing 100084, China.
\IEEEcompsocthanksitem X. Song is with the Department of Electrical \&
Computer Engineering, Portland State University, Portland, OR
97207-0751, USA.}% <-this % stops an unwanted space
}
%\thanks{Manuscript received April 19, 2005; revised September 17, 2014.}}

% note the % following the last \IEEEmembership and also \thanks - 
% these prevent an unwanted space from occurring between the last author name
% and the end of the author line. i.e., if you had this:
% 
% \author{....lastname \thanks{...} \thanks{...} }
%                     ^------------^------------^----Do not want these spaces!
%
% a space would be appended to the last name and could cause every name on that
% line to be shifted left slightly. This is one of those "LaTeX things". For
% instance, "\textbf{A} \textbf{B}" will typeset as "A B" not "AB". To get
% "AB" then you have to do: "\textbf{A}\textbf{B}"
% \thanks is no different in this regard, so shield the last } of each \thanks
% that ends a line with a % and do not let a space in before the next \thanks.
% Spaces after \IEEEmembership other than the last one are OK (and needed) as
% you are supposed to have spaces between the names. For what it is worth,
% this is a minor point as most people would not even notice if the said evil
% space somehow managed to creep in.
}


% The paper headers
% The only time the second header will appear is for the odd numbered pages
% after the title page when using the twoside option.
% 
% *** Note that you probably will NOT want to include the author's ***
% *** name in the headers of peer review papers.                   ***
% You can use \ifCLASSOPTIONpeerreview for conditional compilation here if
% you desire.



% The publisher's ID mark at the bottom of the page is less important with
% Computer Society journal papers as those publications place the marks
% outside of the main text columns and, therefore, unlike regular IEEE
% journals, the available text space is not reduced by their presence.
% If you want to put a publisher's ID mark on the page you can do it like
% this:
%\IEEEpubid{0000--0000/00\$00.00~\copyright~2014 IEEE}
% or like this to get the Computer Society new two part style.
%\IEEEpubid{\makebox[\columnwidth]{\hfill 0000--0000/00/\$00.00~\copyright~2014 IEEE}%
%\hspace{\columnsep}\makebox[\columnwidth]{Published by the IEEE Computer Society\hfill}}
% Remember, if you use this you must call \IEEEpubidadjcol in the second
% column for its text to clear the IEEEpubid mark (Computer Society jorunal
% papers don't need this extra clearance.)



% use for special paper notices
%\IEEEspecialpapernotice{(Invited Paper)}

% for Computer Society papers, we must declare the abstract and index terms
% PRIOR to the title within the \IEEEtitleabstractindextext IEEEtran
% command as these need to go into the title area created by \maketitle.
% As a general rule, do not put math, special symbols or citations
% in the abstract or keywords.

% make the title area
%\maketitle


% To allow for easy dual compilation without having to reenter the
% abstract/keywords data, the \IEEEtitleabstractindextext text will
% not be used in maketitle, but will appear (i.e., to be "transported")
% here as \IEEEdisplaynontitleabstractindextext when the compsoc 
% or transmag modes are not selected <OR> if conference mode is selected 
% - because all conference papers position the abstract like regular
% papers do.
%\IEEEdisplaynontitleabstractindextext
% \IEEEdisplaynontitleabstractindextext has no effect when using
% compsoc or transmag under a non-conference mode.



% For peer review papers, you can put extra information on the cover
% page as needed:
% \ifCLASSOPTIONpeerreview
% \begin{center} \bfseries EDICS Category: 3-BBND \end{center}
% \fi
%
% For peerreview papers, this IEEEtran command inserts a page break and
% creates the second title. It will be ignored for other modes.
%\IEEEpeerreviewmaketitle

\twocolumn[
  \begin{@twocolumnfalse}
\maketitle

\begin{center}
  \begin{tabular}{r l}
    \hline\hline
    & \\
    Manuscript Number: & 15-TIE-3480\\
    Manuscript Title: & Formal Modeling and Verification of a Rate-Monotonic Scheduling Implementation\\
    & with Real-Time Maude \\
    Submitted to: & Transactions on Industrial Electronics \\
    Manuscript Type: & Regular paper \\
    & \\
    \hline\hline
  \end{tabular}
  \bigskip
\end{center}

  \end{@twocolumnfalse}
]

\section{General Response}
We thank all the reviewers for their careful considerations on our
paper. The comments are very insightful and invaluable. Thanks to the
comments, we have improved the paper a lot to make it clearer and
better presented.

The main changes in the paper are as following, which we have also
highlighted in the revised manuscript:
\begin{itemize}
\item
Section~IV about the formal model of the target implementation is
revised. Some technical details have been simplified and more literal
explanations are used to deliver our ideas.

\item
Section~V.B is expanded. Some discussion on the efficiency of our
approach is added.

\item 
The detailed technical proof of Theorem~2 in Section~V.C has been
included in the Appendix.

\item
Section~VI about related work has been improved. 
\end{itemize}

Detailed responses to all the comments are shown in the rest of this
document.


\section{Responses to Reviewer 1}
\subsection{General Comments}

\hide{ The paper shows a realistic implementation of a Rate-Monotonic
  Scheduling algorithm using Real-Time Maude (a modeling language
  based on rewriting logic) by taking into account the overhead of
  scheduling and some other details of the hardware platform. The
  correctness of the implementation with respect to the algorithm and
  the completeness of the model are verified by model checking, and
  validated within different realistic scenarios.}  

\COMMENT 

On the positive side, this paper presents a solid piece of
work. Periodic task scheduling is an important (and difficult)
problem, and the topic of the paper appears to be well suited within
the field of the industrial real-time systems. Overall the paper is
well structured, technically accurate, presents the work in a
comprehensive and reasonable way, and tackles theoretical as well as
practical aspects. There are two principal contributions of the paper:

1) A novel implementation of the Rate-Monotonic Scheduling algorithm,
which contains more complex and realistic details instead of the ideal
setting.

2) Rate-Monotonic Scheduling is investigated using Real-Time Maude for
the first time, to apply formal methods such as model checking and
theorem proving to analyze theoretical results, verify desired
properties, and evaluate results.

\ANSWER 

Thanks for the pros. We are happy that the work in the paper interests
the readers. We hope that it provides a new and formal way to verify a
real-time system, which is able to investigate important problems
(such as periodic task scheduling) in a more realistic level.

\COMMENT 

The weak point is the lack of evidence of practicality, efficiency and
scalability. Since there are only small and simplistic scenarios to
analyze this model, it is difficult to judge the effectiveness of the
proposed implementation. The key scenarios appear conveniently small
to allow the implementation fully supports all these examples,
returning at the same time appropriate counterexamples to guide and
adjust the design. Is this true for larger examples of "real world"
applications within the field of industrial real-time systems? How do
the authors propose to handle very large scenarios?  How does it
behave on typical real-time scheduling problems used in industry? I
would like to see more complex examples to convince the reader of the
practicability of the approach on the assumptions given for the model,
and would probably increase the mentioned benefits. The paper does not
give any directions to the above problems, and some sentences pointing
out the difficulties in the implementation of more generic and
realistic scenarios would be most welcome to understand the real
contribution of the paper.

\ANSWER 

Thanks for the suggestions. As mentioned in the paper, the system
under verification is a real-world RMS implementation, which is used
in an avionic control system. The scenarios presented in Section V.B
are also real scenarios from the online system. Our industrial partner
has very strict timing requirements. They use at most 5 tasks for
scheduling to avoid high overhead. They agree that the verification
results presented in the paper satisfy their practical requirements.

On the other hand, as presented in the revised Section V.B, we have
examined our approach by verifying randomly generated scenarios where
the number $n$ of tasks are over $10$. In that case, the least common
multiple of the periods of all tasks can be as large as $10^5$ times
the interrupt cycle $T$. All feasible combinations of execution traces
are checked in our approach. Furthermore, the number $n$ of tasks can
be $20$ and even more if we allow different tasks possess a same
period.

\hide{
the modeled RMS implementation serves in an avionic control systems
from our industrial partner. It is a real-world application within the
field of industrial real-time systems. And the scenarios presented in
Section V.B are also real scenarios that our partner is using in the
system. Our partner indicated that there would be at most $5$ tasks in
the task set for the implementation. On the other hand, in the current
version, we have randomly generated large numbers of test cases to
examine the efficiency of our schedulability test that is based on
model checking, as discussed in Section V.B. The capability of model
checking is restricted by the scale of the state space of the
model. In our model, the model checking can handle scenarios where
$mn$ is up to $10^6$ in an acceptable period of time. With $mn=10^6$
and reasonable task periods, the number $n$ of tasks can be at least
$10$, which has already fulfilled the requirements of the target
implementation. Further with particular task periods, the number $n$
of tasks can be $20$ or more.
}

\COMMENT 

One of my biggest problems with the theoretical part of this work is
that the paper is technically solid as far as I can tell, but there's
not much yet in the way of theorems and results about formal
properties in Maude. The complete and detailed proof of the Theorem 2
should be provided somewhere. If the proof of the main results of
schedulability and correctness of the model is straightforward (as is
suggested) it seems not necessary to present a complex formalism on
Real-Time Maude and all the theorems provided are natural consequences
of Theorem 1. I agree that even non-surprising results need to be
investigated in the proof of Theorem 2, some sentences pointing out
the difficulties in the theoretical formalization in Real-Time Maude
would be most welcome. Please, include (e.g., in an appendix or even
in a technical report) a detailed proof.

\ANSWER 

We are happy that the readers are interested in the detailed technical
proof, and we have now included it in the Appendix. It is true that
readers may be interested in the detailed proof of Theorem~2 to feel
convinced of the completeness of our approach. However, the proof
needs a bit more theoretical background about rewriting logic and more
technical details of our model, which we have simplified in the paper
in order to ease the understanding of our approach. We would prefer to
refer the detailed proof to a non-anonymous technical report, if
possible, in the final version.

\COMMENT 

Another important point which needs improvement is the description of
the implementation. I guess other readers might have less difficulties
with the motivations if they know Maude and the Real-Time Maude
extension, but if you aim for a more general audience, it might not
work. Many pages are spent for the description of the formal modeling
of the implementation and the section is hard to read. I highly
recommend the authors to thoroughly revise this part of the paper to
make it easy-to-read. It would be better preferable to have a much
shorter description of the principal insights for the implementation
followed by more convincing examples and benchmarks with respect to
related work to convince the reader of the practicability of the
approach. You should try to summarize this section and to extend the
original contribution. From my point of view, there is a trade off in
this paper between the contributions and the presentation of the
implementation.

\ANSWER 

Thank you for the invaluable suggestions. Section IV has been
revised. Some unnecessary detailed definitions presented as code are
removed, and replaced by more literal explanations. We hope that the
current presentation would make Section IV more easy-to-read, even for
an audience who did not know Real-Time Maude before. On the other
hand, Section VI (Related Work) is also improved. We compare our work
with the existing theoretical work and the existing verification work
in a more complete way.


\subsection{Detailed Comments}

\COMMENT 

Page 7, Related Work. More comparisons with other approaches based on
model checking would help to evaluate pros and cons of an
implementation with Real-Time Maude with respect to different
languages and tools. The authors should review the comparison with
[24] and [25] more thoroughly. You should benchmark your approach
against others.

\ANSWER 

Thanks for the pointing this out. The related work section is now
improved. More comparisons with the theoretical approaches and with
[24,25] are added. The differences with [24,25] in the models and in
the way of modeling are mainly discussed. On the other hand, it is a
bit difficult to compare the efficiency and scalability between
[24,25] and our approach. One reason is that we have a different
setting with [24] and [25], since the objectives are different: we aim
at verifying an RMS implementation from a real-world real-time system,
while [24] and [25] targeted the RMS algorithm itself. Another reason
is that [24] and [25] presented no discussion on the efficiency and
scalability. Furthermore, they used \verb|TMSVL| and an extension of
SPIN, respectively, as verification tools, which are not open-source,
making us unable to re-implement their work.

\COMMENT 

Page 8, Conclusions. Where in the paper do you show that the details
of your "realistic" implementation are "sufficient" for the behaviors
of all real systems used in the current industry? This should be
briefly discussed.

\ANSWER 

Sorry for the confusion. The statement has been corrected. The work
presented in the paper aims at modeling and verifying a realistic RMS
implementation in an industrial avionic system. It is a piece of real
work in the industry. In this sense, ``sufficient'' means that the
details described in our model meet the requirements and expectations
from the users and the engineers. On the other hand, we believe that
our approach could be applied to other similar systems as well.


\section{Responses to Reviewer 2}
\subsection{General Comments}

\COMMENT 

One of the problems with this paper is that it is not completely
self-contained.  Some knowledge about Real-Time Maude is necessary in
order to fully understand it.  I found the short introduction about
Real-Time Maude (section II.B) not enough for understanding the
formalism used in the rest of the paper (see details below).

\ANSWER 

Thank you very much for pointing this out. The Sections II.B and IV
are improved to be more self-contained now. The idea is that, Section
II.B presents mainly an overview of Real-Time Maude, and then Section
IV describes the model with introducing necessary syntax on-the-fly.

\COMMENT 

Another concern is that the significance of the paper could be much
better if the paper addressed the problem of modelling RMS in a more
general way. Instead, the paper takes the form of a case study, where
the formal modelling of one specific RMS implementation is presented,
rather than a general method for modelling and analyzing RMS
implementations.

\ANSWER 

Thank you for the suggestion. Yes, we agree that we have two choices
to do the work presented in the paper: one is to develop a general way
to model RMS algorithms or implementations, and then instantiate the
general model to get a concrete one for our target system; the other
is to shape an accurate model of the target system directly.

We chose the latter according to the following considerations. By
experience, \emph{the development engineers (who write the code) and
  verification engineers (who verify the code or the system) are
  usually not the same persons.} When a verification engineer models a
system, he is actually abstracting the system or the code. It is
difficult to ensure that the model behaves the same as the system or
the code.  And it is more difficult to make other engineers and users
believe that the model behaves the same as the system or the
code. Furthermore, the developers and users hope that the model
behaves not only the same as what they had in mind, but also the same
as what is written in the code. A way out is to make the model not
only \emph{behave like} the code, but also \emph{``look'' like} the
code. This requires the model to be as accurate as possible, based on
a fact that the expressiveness of the modeling language is powerful
enough. It makes the model specific, but improves the engineers'
confidence in the model and the verification results. That is also why
we emphasize the corresponding relationship between the functions in
the model and the lines in the pseudocode, when we introduce our model
in Section~IV.

On the other hand, we are also interested in the first direction. This
will be the future work and we would try to get a better balance
between them.

\COMMENT 

Also, the discussion about soundness and completeness of the analysis
(section V.C) is not fully developed and, once again, it refers to the
specific model rather than being a general result.

\ANSWER 

Thanks for this comment. The detailed proof of Theorem~2 is now
included in the Appendix. On the other hand, since we chose the
accurate way to achieve the model as discussed above, yes, the proof
is a specific one.


\subsection{Detailed Comments}

\COMMENT 

[Section II.B] In the definition of $IR$, it is not clear what $s$ and
$s'$ are. I guess they are states, but the concept of state was not
introduced.

\ANSWER 

Yes, they can be states. But more generally, they are terms. This part
has been modified to specify $s$ and $s'$ (in fact, $t$ and $t'$ now)
clearly.

\COMMENT 

[Section II.B] In the definition of a class, the authors should
specify that a class includes a collection of rules, otherwise this
fact may be missed.

\ANSWER 

Thanks very much for reminding. This is added now.

\COMMENT 

[Section III] Page 2 right column, line -24: The implementation is
shown as schedule() in Figure 1 $\rightarrow$ The pseudocode of
schedule() is shown in Figure 1

\ANSWER 

Thanks for the careful reading. It is done now.

\COMMENT 

[Section IV.A] I don't understand the meaning of \verb|[ctor]| in the
definition of some operations (maybe because I have no specific
background on Maude). I could not find the explanation of this writing
in the introduction to Maude.

\ANSWER 

Sorry for missing the explanation of \verb|[ctor]|. It is now added
when \verb|[ctor]| is used for the first time.

\COMMENT 

[Section IV.A] For the stack sort, the authors mention operations
push, pop and peek, but then you don't write the definitions of these
operations. Instead, they write the definitions of bottom and \#,
which are not explained in the text. This part should be made clearer.

\ANSWER 

Sorry for the misleading presentation in the text. In fact,
\verb|bottom| and \verb|#| are constructors of sort \verb|Stack|. This
part is modified now. Some detailed definitions in code are removed,
and are replaced by more explanations. We hope that the current
presentation would make Section IV more easy-to-read.

\COMMENT 

[Section IV.A] Also, the meaning of NzNat is not explained.

\ANSWER 

Sorry for this incompleteness. \verb|NzNat| meant non-zero natural
numbers. \verb|NzNat| is now replaced by \verb|Nat| to reduce
unnecessary complexity for the readers.

\COMMENT 

[Section V.C] In the statement of Theorem~1, the definition of
time-robust real-time rewrite theory is missing. Also, the meaning of
tick-stabilizing atomic propositions is not defined. This makes the
section not self-contained.

\ANSWER 

Sorry for the inconvenience for reading. In fact, the definitions of
time-robustness and tick-stabilization (which is now replaced with
tick-invariance) require more knowledge about rewriting logic. We
avoid introducing the accurate definitions of them. Instead, now we
give short descriptions of them before Theorem~1. More detailed
introduction can be found in the Appendix.

\COMMENT 

[Section V.C] The proof of theorem 2 is missing. The authors just
indicate how the proof could be developed, but they do not develop
it. A possibility would be to point to a document where this proof has
been developed.

\ANSWER 

We are very happy that the readers are interested in the detailed and
technical proof, which is now included in the Appendix.  The detailed
proof requires a bit more theoretical background about rewriting logic
and more details of our model, which we tried to simplify in the paper
in order to ease the understanding of our approach. We would prefer to
refer the detailed proof to a non-anonymous technical report, if
possible, in the final version.


\section{Responses to Reviewer 3}

\COMMENT 

RMS has many different realization methods, no comparison is given to
justify the strong point of real-time maude when applied to RMS.

\ANSWER 

The comparison could be found in Section VI (and we have expanded
according to the reviewers' comments :-)). We would like to clarify
that we are \emph{not implementing} RMS, we are actually
\emph{verifying} it. We choose Real-Time Maude since it verifies the
model using model checking technique. If verification passes, no
deadline will be missed in the implementation as long as the model
assumptions are met.

\hide{
The comparison is given in the related work section (Section
VI). We would say that we are not implementing RMS algorithm in
Real-Time Maude. In fact, we are using Real-Time Maude to model an
realistic implementation of RMS from an industrial avionic
system. There is no theoretical reason telling us that we have to use
Real-Time Maude to model the RMS implementation. However, the existing
verification work as far as we know cannot fit our needs, as discussed
in Section VI. Therefore, we developed the work in the paper, as a
piece of real work in the industry.
}

\COMMENT 

Rate-monotonic scheduling is a very simple scheduling and has been
sufficiently investigated in the field of embedded system, the authors
claim the innovation of this paper as the implementation using
real-time maude, however, I cannot see any special point (or
advantage) when using this real-time maude in modelling RMS.

\ANSWER 

Thank you for the insightful comment. We agree that RMS has been
intensively studied because of its importance. However, most of the
results are based on the assumption that \emph{task switching does not
  take time}. That assumption is not realistic. Models with and
without the assumption have totally different behaviors. For instance,
as shown in Figure~2(b), in the RMS implementation we verified, the
switching from time $9$ to $11$ blocks the interrupt handling, hence
delaying the initiation time of $\tau_1$. This kind of phenomenon, to
our knowledge, would not happen in the theoretical analysis
models. Although there is very few results that take into account the
switching overhead (as discussed in Section VI), none of them is
generic and our system under verification is not in their scope.

Despite the above reason, the most important motivation of this paper
is to verify the system not only for the schedulability, but also for
the correctness. That is, we want to verify that the target system
does schedule the tasks under the RMS algorithm. The work presented
should be seen as a verification problem, instead of a schedulability
test. From verification point of view, this paper models a real-world
RMS implementation with sufficient technical details. Two important
properties--schedulability and correctness--are verified by model
checking technique, and the soundness and completeness of the results
are demonstrated.

\hide{
is a very simple scheduling and is the optimum at the same time among
the fixed-priority ones, making it widely applied and sufficiently
investigated. Besides the ideal setting, the schedulability of RMS was
also investigated under different kinds of implementations with
consideration about overhead, as we discuss in the related work
section (Section VI). Despite the reason that the existing results do
not fit our target system using RMS, the most important motivation to
develop the work presented is that, we aim at verifying the system not
only for the schedulability, but also for the correctness. That is, we
want to verify that the target system does schedule the tasks under
the RMS algorithm. The work presented should be seen as a verification
problem, instead of a schedulability test.
}

\COMMENT 

The assumptions given in IV is too rigid, which makes the scheduling
problem to be investigated by very simple, then what is the
difficulty?

\ANSWER 

Thank you for the question. When model checking technique is used to
verify a given system, the tool (model checker) explores the whole
system state space to check whether the given property holds. All
feasible combinations of execution traces are checked. As a result,
the state space explosion is the most common problem for a model
checker. That means the state space of the system/model is too huge
for the tool to explore with acceptable temporal and spatial cost. The
scale of the state space depends on the complexity of the model of the
system.

As an industrial verification application, we are not trying to make
the problem as difficult as possible. Instead, we are trying to
simplify the model so that the existing tools are able to solve, while
the target system satisfies the assumptions on the model so that the
verification results are trustworthy.

On the other hand, the assumptions are actually not that rigid. For
example, the mask mechanism considered makes the job initiation times
of task $\tau_i$ possibly not equal $kT_i$, implying that the
deadlines of the jobs may be not equal to $(k+1)T_i$ (see assumption
A1'). This is also discussed in the beginning of Section IV, pointing
out the blocking at time $10$ in Figure~2(b). Another example is that
non-determinism exists in our model, as discussed in Section~VI.

\COMMENT 

The description in section IV (A-E) is trivial, it can only be viewed
as a simple application of the rewriting logic.

\ANSWER 

Thanks for the careful reading. We agree that this is an application
of rewriting logic. In this paper, we do apply the existing theories
and techniques of rewriting logic to solve a real industrial problem,
which has not yet been investigated by this sort of techniques,
achieving an industrial contribution.

\COMMENT 

The model checking part is not sufficiently discussed, the authors
only list the commands but not illustrate the mechanism behind these
commands.

\ANSWER 

Thank you for the suggestion. Our work does focus on how to model a
realistic industrial system and how to apply formal techniques (such
as model checking) to verify important properties of the system. The
mechanism behind the model checking commands (i.e. of a model checker)
is another issue.

The field of model checking techniques and the implementation of model
checkers have been intensively studied for decades. If the reviewer is
interested in the mechanism of the model checker provided by Real-Time
Maude, we would recommend the reference [A] to the reviewer:

\noindent
[A] S. Eker, J. Meseguer, and A. Sridharanarayanan, ``The Maude LTL
Model Checker'', {\it Electr. Notes Theor. Comput. Sci.}, vol. 71,
pp. 162-187, 2002.

\hide{
However, since the model checker is not part of our contributions in
the paper, we do not think it is appropriate to illustrate the
mechanism behind the model checking commands, which may confuse the
readers on the other hand.}


\hide{
% use section* for acknowledgment
\ifCLASSOPTIONcompsoc
  % The Computer Society usually uses the plural form
  \section*{Acknowledgments}
\else
  % regular IEEE prefers the singular form
  \section*{Acknowledgment}
\fi

The authors would like to thank...
}

% Can use something like this to put references on a page
% by themselves when using endfloat and the captionsoff option.
\ifCLASSOPTIONcaptionsoff
  \newpage
\fi



% trigger a \newpage just before the given reference
% number - used to balance the columns on the last page
% adjust value as needed - may need to be readjusted if
% the document is modified later
%\IEEEtriggeratref{8}
% The "triggered" command can be changed if desired:
%\IEEEtriggercmd{\enlargethispage{-5in}}

% references section

% can use a bibliography generated by BibTeX as a .bbl file
% BibTeX documentation can be easily obtained at:
% http://www.ctan.org/tex-archive/biblio/bibtex/contrib/doc/
% The IEEEtran BibTeX style support page is at:
% http://www.michaelshell.org/tex/ieeetran/bibtex/
%\bibliographystyle{IEEEtran}
% argument is your BibTeX string definitions and bibliography database(s)
%\bibliography{IEEEabrv,submission}
%
% <OR> manually copy in the resultant .bbl file
% set second argument of \begin to the number of references
% (used to reserve space for the reference number labels box)


% biography section
% 
% If you have an EPS/PDF photo (graphicx package needed) extra braces are
% needed around the contents of the optional argument to biography to prevent
% the LaTeX parser from getting confused when it sees the complicated
% \includegraphics command within an optional argument. (You could create
% your own custom macro containing the \includegraphics command to make things
% simpler here.)
%\begin{IEEEbiography}[{\includegraphics[width=1in,height=1.25in,clip,keepaspectratio]{mshell}}]{Michael Shell}
% or if you just want to reserve a space for a photo:

\hide{

\begin{IEEEbiography}{Jiaxiang Liu}
Biography text here.
\end{IEEEbiography}


% if you will not have a photo at all:
\begin{IEEEbiographynophoto}{John Doe}
Biography text here.
\end{IEEEbiographynophoto}

% insert where needed to balance the two columns on the last page with
% biographies
%\newpage

\begin{IEEEbiographynophoto}{Jane Doe}
Biography text here.
\end{IEEEbiographynophoto}
}

% You can push biographies down or up by placing
% a \vfill before or after them. The appropriate
% use of \vfill depends on what kind of text is
% on the last page and whether or not the columns
% are being equalized.

%\vfill

% Can be used to pull up biographies so that the bottom of the last one
% is flush with the other column.
%\enlargethispage{-5in}



% that's all folks
\end{document}



% This is LLNCS.DOC the documentation file of
% the LaTeX2e class from Springer-Verlag
% for Lecture Notes in Computer Science, version 2.4
\documentclass{llncs}
\usepackage{alltt}
\usepackage{amsmath}
\usepackage{amsfonts}
\usepackage{amssymb}
\usepackage{mathabx}
\usepackage{algorithm}
\usepackage{algpseudocode}
\usepackage{enumerate}
\usepackage{epic}
\usepackage{pstricks}
%\usepackage{llncsdoc}

\newcommand{\hide}[1]{\ignorespaces}
\newcommand{\jx}[1]{{\bf Jiaxiang: }#1{ \bf End}}

\begin{document}
%\thispagestyle{empty}
%\begin{flushleft}
%\end{flushleft}

\title{Formal Modeling and Verification of a Rate-Monotonic Scheduling Implementation with Real-Time Maude}
\author{Jiaxiang Liu\inst{1,2}}
\institute{School of Software, Tsinghua University, Beijing, China
  \and \'Ecole Polytechnique, Palaiseau, France}
\maketitle
\thispagestyle{empty}

\section{Introduction}
Periodic task scheduling is one of the most important topics within
the field of real-time systems, due to the large number of control
systems that require cyclic activities~\cite{buttazzo2011hard}. A set
of periodic tasks is said to be \emph{schedulable} with respect to
some scheduling algorithm if all jobs meet their
deadlines. \emph{Rate-Monotonic Scheduling} (\emph{RMS}) is a
\emph{fixed} priority scheduling algorithm for preemptive hard
real-time environments proposed by Liu and
Layland~\cite{DBLP:journals/jacm/LiuL73}, which assigns priorities to
jobs according to the periods of the corresponding tasks: the smaller
period, the higher priority. RMS is proved to be the \emph{optimal}
fixed priority scheduling algorithm~\cite{DBLP:journals/jacm/LiuL73},
in the sense that any set of tasks, which is schedulable under
\emph{some} fixed priority scheduling algorithm, is also schedulable
with respect to RMS. It is widely used in safety-critical real-time
applications, such as trains and avionics, thanks to its optimality
and easiness to implement.

Liu and Layland~\cite{DBLP:journals/jacm/LiuL73} gave a sufficient
condition for the schedulability of a set of $n$ tasks scheduled by
RMS: $\displaystyle\Sigma^n_{i=1}C_i/T_i \le n(2^{1/n}-1)$, where
$C_i$ and $T_i$ are the \emph{computation (time) requirement} and the
period of task $\tau_i$, respectively. Two main directions on RMS have
been explored since then. One is to relax the assumptions on the
original RMS model, making it applicable on more systems.  For
instance,
\cite{DBLP:conf/rtss/LehoczkySS87,DBLP:journals/rts/SpruntSL89,DBLP:conf/rtss/LehoczkyR92,DBLP:journals/tc/StrosniderLS95}
allow aperiodic tasks in the scheduling,
\cite{DBLP:journals/pe/LeungW82,audsley1993deadline} generalize RMS to
be \emph{deadline-monotonic}, \cite{DBLP:journals/tc/ShaRL90} allows
resource sharing among tasks,
\cite{dhall1978real,DBLP:journals/rts/LopezGDG03,DBLP:journals/tpds/LopezDG04,DBLP:journals/tc/BaruahG03}
extend RMS on multiprocessors, and
\cite{DBLP:journals/rts/OhS94,DBLP:journals/rts/GhoshMMS98,DBLP:journals/tpds/BertossiMR99}
enhance fault-tolerance. The other direction is to generate better
schedulablity test conditions for the algorithm and its
extensions~\cite{DBLP:conf/rtss/LehoczkySD89,DBLP:conf/rtss/KuoM91,DBLP:journals/tc/BiniBB03,DBLP:journals/rts/LopezGDG03,DBLP:journals/tc/BaruahG03,gardner1999}. The
RMS algorithm is no doubt of practical importance.

Extensive effort to apply formal methods, such as model checking and
theorem proving, has been made to analyze safety-critical systems for
the past few
years~\cite{DBLP:journals/iandc/MeseguerR13,DBLP:journals/cacm/Leroy09,DBLP:conf/sosp/KleinEHACDEEKNSTW09}. However,
as far as we know, few~\cite{DBLP:conf/iceccs/CuiDT14,TianD2011}
attempt to analyze the RMS algorithm, while no work for
implementations of RMS is found.

In this paper, we use \emph{Real-Time Maude}, a \emph{rewriting}-based
modeling language and analyzing tool for real-time systems, to model a
realistic implementation of RMS and then verify some desired
properties on the built model. Based on a realistic implementation,
our model extends the standard one proposed
in~\cite{DBLP:journals/jacm/LiuL73}, by allowing computation
requirements of tasks to be non-constant, and by considering overhead
and other details of the enviroments.

The rest of this paper is organized as
follows. Section~\ref{s:background} gives some background of the RMS
algorithm and Real-Time Maude.  Section~\ref{s:imp} presents the RMS
implementation that we model and analyze.  Section~\ref{s:formalism}
introduces how we model the RMS implementation using Real-Time
Maude. Then Section~\ref{s:verification} explains how to verify the
desired properties and to evaluate the results. Related work is
discussed in Section~\ref{s:relate}. We conclude the paper in
Section~\ref{s:conclusion}.

\section{Background}
\label{s:background}

\subsection{Rate-Monotonic Scheduling Algorithm}
\label{ss:rms}
%\subsection{Formulation of the Standard Setting}
A task set consists of \emph{only} $n$ periodic tasks
$\tau_1,\ldots,\tau_n$. Each task $\tau_i$ has a period $T_i$ and a
computation requirement $C_i$. First jobs of all tasks are assumed to
be initiated simultaneously. This means that jobs corresponding to
task $\tau_i$ are initiated at times $kT_i$ with integers $k\ge
0$. Deadlines consist of runability constraints only: the job
initiated at time $kT_i$ has $(k+1)T_i$ as its deadline, which is the
initiation time of the next job. The RMS algorithm chooses the
labeling such that $T_1\le T_2\le \ldots \le T_n$. Consequently,
$\tau_i$ receives priority $i$, assuming smaller numbers have higher
priorities. The following assumptions are made:

(A1) Computation requirement $C_i$ for each task $\tau_i$ is constant
and does not vary with time.

(A2) Tasks are independent, such that they are ready to run at their
initiation times and can be preempted instantly (ignoring all
blocking).

(A3) All overhead, such as task switching times, is ignored.

However in this paper, we consider an implementation instead of the
RMS algorithm itself, thus the model would be more complicated than
this standard, ideal setting.  Assumptions~(A1) and (A3) will be
relaxed later to obtain a more realistic analysis model.

\subsection{Real-Time Maude}
Based on \emph{rewriting logic}~\cite{DBLP:journals/jlp/Meseguer12},
\emph{Real-Time Maude} is a language and tool that extends
\emph{Maude} to support the formal specification and analysis of
real-time systems~\cite{DBLP:journals/lisp/OlveczkyM07}.

\subsubsection{Specification}
Real-Time Maude models systems as \emph{modules}. A module specifies a
\emph{real-time rewrite theory} ${\cal R} = (\Sigma, E\cup A , IR,
TR)$, where:
\begin{itemize}
\item $\Sigma$ is an algebraic \emph{signature}, that is, a set of
  declarations of \emph{sorts}, \emph{subsorts} and \emph{function
    symbols}. The function symbols are allowed to be mixfix.
\item $(\Sigma, E\cup A)$ is a \emph{membership equational logic
  theory}, with $E$ a set of conditional equations on $\Sigma$, and
  $A$ a set of equational axioms such as associativity, commutativity
  and identity.  $(\Sigma, E\cup A)$ models the system's ``static''
  states as an algebraic data type, and includes a built-in
  specification of a sort \verb|Time|.
\item $IR$ is a set of \emph{labeled conditional rewrite rules}
  specifying the system's local transitions. Each rule has the form
  $[l]~:~t\rightarrow t'\mbox{ \textbf{if} }\bigwedge^n_{j=1}cond_j$,
  where each $cond_j$ is an equality $u_j=v_j$ and $l$ is a
  \emph{label}. Such a rule specifies an \emph{instantaneous
    transition}, without consuming time, from an instance of $t$ to
  the corresponding instance of $t'$, \emph{provided} the conditions
  hold.
\item $TR$ is a set of (\emph{labeled}) \emph{tick rules}
  $[l]~:~\{t\}\rightarrow\{t'\} \mbox{ \textbf{in time} }r\mbox{
  \textbf{if} }cond$ that specify \emph{timed transitions} advancing
  time in the \emph{entire} state $t$ by $r$ time units. $IR$ and $TR$
  together model the ``dynamic'' behaviors of the system.
\end{itemize}

In rewriting logic, rewrite rules are applied non-deterministically,
that is, when several rules can be applied on a given $t$, any of them
may be chosen. Hence non-deterministic behaviors can be modeled
naturally in Real-Time Maude.

Real-Time Maude also supports specifications in \emph{object-oriented}
style.  A class declaration $\texttt{class }C\texttt{ |
}att_1\texttt{:}s_1\texttt{,}\ldots\texttt{,}att_n\texttt{:}s_n$
defines a class $C$ with attributes $att_1$ to $att_n$ of sorts $s_1$
to $s_n$, respectively. An \emph{object} of class $C$ is represented
as a term $\texttt{< } O\texttt{:} C \texttt{ | }
att_1\texttt{:}val_1\texttt{,} \ldots
\texttt{,}att_n\texttt{:}val_n\texttt{ >}$ of sort \verb|Object|,
where $O$, of sort \verb|Oid|, is the object's \emph{identifier}, and
$val_1$ to $val_n$ are the values of the attributes $att_1$ to
$att_n$. A \emph{subclass} inherits all the attributes and rules of
its superclasses.

\subsubsection{Formal Analysis}
Real-Time Maude provides many useful commands and tools to analyze a
model. For example, \verb|rewrite| allows to execute the model,
symbolically; given an initial state, \verb|search| is used to search
reachable states satisfying the desired properties; the Maude's
\emph{Inductive Theorem Prover} (ITP) can be applied to interactively
prove properties written in \emph{membership equational logic}. 

In this paper, we only consider Real-Time Maude's \emph{linear
  temporal logic model checker}, which analyzes whether \emph{each}
behavior satisfies a temporal logic formula. \emph{State propositions}
are terms of sort \verb|Prop|. Their symantics is defined by equations
of the form $\texttt{ceq } statePattern \texttt{ |= } prop \texttt{ =
} b \texttt{ if } cond$, with $b$ a term of sort \verb|Bool|, stating
that $prop$ evaluates to $b$ in states which are instances of
$statePattern$ provided the condition $cond$ holds. These equations
together define $prop$ to hold in all states $t$ that make $t \texttt{
  |= } prop$ evaluate to \verb|true|. A temporal logic \emph{formula}
is constructed by state propositions and temporal logic operators such
as \verb|True|, \verb|False|, \verb|~|(negation), \verb|/\|,
\verb|\/|, \verb|->|(implication), \verb|[]|(``always''),
\verb|<>|(``eventually'') and \verb|U|(``until''). Real-Time Maude
supports both \emph{timed} and \emph{untimed LTL model checking}. The
untimed model checking command
\begin{alltt}
  (mc \(t\) |=u \(formula\) .)
\end{alltt}
checks whether the temporal logic $formula$ holds in all behaviors
starting from the initial state $t$, \emph{with no time limit}.


\section{The Implementation of RMS}
\label{s:imp}
The implementation is in a realistic avionic operating system.
Interrupts would be triggered by the clock every $T$, which we call
\emph{interrupt cycle}. When an interrupt request occurs, if the
system is interruptable, i.e. the interrupt mask is cleared, the
handler function $schedule()$ will be invoked; otherwise $schedule()$
will be pending until the interrupt mask becomes cleared.  The
implementation is shown as $schedule()$ in Algorithm~\ref{a:schedule},
where $taskList$ is the list of periodic tasks to be scheduled. We
assume that the list is ordered descendingly by priorities, and both
variables $taskList$ and $timer$ are global. In this implementation,
there is only one kind of interrupt, the period $T_i$ of each task is
a multiple of $T$, and the tasks are independent, meeting
assumption~(A2).

\begin{algorithm}
  \caption{The C-Like Pseudocode of $schedule()$}
  \label{a:schedule}
  \begin{algorithmic}[1]
\Function{$schedule$}{$ $}{}
  \State \Call{$int\_o\!f\!\!f$}{$ $}; \Comment{to disable interrupts} \label{l:1stline}
  \State \Call{$updateStatus$}{$taskList$}; \label{l:updatestatus}
  \State $timer = timer + 1$; \label{l:timer} \label{l:inc}
  \State $p = taskList$;
  \While{$p$} \label{l:startrun1st}
    \If{$p\rightarrow status == \textit{INTERRUPT}$}
      \State \Return;
    \ElsIf{$p\rightarrow status == \textit{READY}$}      
      \State $p\rightarrow status = \textit{RUNNING}$;
      \State \Call{$int\_on$}{$ $}; \Comment{to enable interrupts} \label{l:endrun1st}
      \State $p\rightarrow function()$; \Comment{to execute the task} \label{l:function}
      \State \Call{$int\_o\!f\!\!f$}{$ $};
      \State $p\rightarrow status = \textit{DORMANT}$;
    \EndIf
    \State $p = p\rightarrow next$;
  \EndWhile
\EndFunction
\Function{$updateStatus$}{$p$}
  \While{$p$}
    \If{$p\rightarrow status == \textit{RUNNING}$} \label{l:startupdate}
      \State $p\rightarrow status = \textit{INTERRUPT}$;
    \EndIf
    \If{$timer~\%~(p\rightarrow period) == 0$} \Comment{the task should be initiated}
      \If{$p\rightarrow status == \textit{DORMANT}$} \Comment{the previous job finishes} 
        \State $p\rightarrow status = \textit{READY}$;
      \Else \Comment{the status is \textit{READY} or \textit{INTERRUPT}}
	\State \Call{$reportTaskError$}{$p$}; \Comment{the task misses its deadline}
      \EndIf
    \EndIf \label{l:endupdate}
    \State $p = p\rightarrow next$;
  \EndWhile
\EndFunction
  \end{algorithmic}
\end{algorithm}

In Algorithm~\ref{a:schedule}, the handler function $scheduler()$
first updates status of all tasks in $taskList$ via function
$updateStatus()$. Then it traverses the list to execute the ready
tasks one by one, or to do a return when encountering an
interrupted\footnote{Note that the status \textit{INTERRUPT} indicates
  the task is interrupted for the moment, or was interrupted before
  but its execution is not complete yet.} task. The function
$updateStatus()$ updates each task in two steps: firstly, if the task
is running, it becomes interrupted; secondly for the task at its
initiation time, if its previous job is complete, it would be set to
be ready, otherwise it misses its deadline, producing an error. Notice
that $schedule()$ is invoked only when the periodic interrupt is
handled. Its execution cannot be interrupted when it is updating
status of tasks or finding the next task to execute, however, it can
be interrupted while executing some task (Line~\ref{l:function}). This
allows the execution of $schedule()$ to be nested.

In the rest of this paper, we use \emph{scheduling} to refer to the
stage, from the moment when a pending interrupt request is detected,
to the moment when the first should-be-run periodic task starts
executing, i.e. Line~\ref{l:function} in
Algorithm~\ref{a:schedule}. Therefore, \emph{scheduling time} mainly
consists of three parts: (i) the time for switching context from the
running task, possibly none, to $schedule()$ when an interrupt request
is handled, (ii) the time spent by $schedule()$ finding and setting
the \emph{first} should-be-run periodic task
(Lines~\ref{l:1stline}-\ref{l:endrun1st} in
Algorithm~\ref{a:schedule}), and (iii) the time for switching context
from $schedule()$ to that first should-be-run task. \emph{Switching}
refers to the stage, from the moment when a periodic task completes
its execution, to the moment when the next should-be-run periodic task
starts executing. \emph{Switching time} thus also consists of three
parts: (i) the time for switching context from the finished task back
to $schedule()$, (ii) the time spent by $schedule()$ finding and
setting the \emph{next} should-be-run periodic task, and (iii) the
time for switching context from $schedule()$ to that next
should-be-run task.

\section{Formal Model of the Implementation}
\label{s:formalism}
We model the implementation under the following assumptions:

(A1') Computation requirement $C_i$ for each task $\tau_i$ is not
constant. It varies randomly between minimum $C^{min}_i$ and maximum
$C^{max}_i$, i.e. within an interval $[C^{min}_i, C^{max}_i]$.

(A2) Tasks are independent, such that they are ready to run at their
initiation times and can be preempted instanly.

(A3') Scheduling time and switching time are considered, while other
overhead is ignored.

These assumptions make our model different from the standard one. For
instance, if an interrupt request occurs while switching is performed,
jobs corresponding to task $\tau_i$ may not be initiated at exactly
$kT_i$. They should wait until switching finishes and the interrupt
mask is cleared. Note that (A2) says that jobs are ready to run at
their initiation times, however, no one can start running at exactly
its initiation time, because scheduling takes time.

\jx{TODO. A figure is needed here to demonstrate the scheduling under
  the three assumptions above.}

In this section, we introduce first how we model the states--the
static aspect--of the system using algebraic data types, then how we
specify the essential behaviors--the dynamic aspect--using rewrite
rules. In particular, modeling of instantaneous behaviors would be
explained in Sections~\ref{ss:ir}, \ref{ss:init}
and~\ref{ss:inthandling}, followed by timed behaviors in
Section~\ref{ss:timedbehavior}.

\subsection{Basic Data Types}
In our model, tasks are identified by their indexes of sort \verb|Nat|
in the $taskList$. We define a sort \verb|MaybeNat| wrapping
\verb|Nat|s to refer to some task, with \emph{constructor} \verb|some|
followed by a \verb|Nat| $n$ indicating the task indexed $n$, and
\verb|none| for no task:
\begin{verbatim}
  op none : -> MaybeNat [ctor] .
  op some_ : Nat -> MaybeNat [ctor] .
\end{verbatim}

A sort \verb|Stack| is introduced to simulate the stack of the system,
storing the tasks that are being interrupted, and equipped with
operations \verb|push|, \verb|pop| and \verb|peek| on it.
\begin{verbatim}
  op bottom : -> Stack [ctor] .
  op _#_ : Nat Stack -> Stack [ctor] .
\end{verbatim}

We also need a sort \verb|Interval| to denote computation requirements
of tasks, and a sort \verb|Counter| to record execution of tasks.  We
call \emph{execution time} the time how long a task has been executing
for. A \verb|Counter| records the execution time and the computation
requirement of a task.
\begin{verbatim}
  op [_,_] : Time Time -> Interval [ctor] .  
  op [_/_] : Time Interval -> Counter [ctor] .
\end{verbatim}

\hide{
At last, to make our model checkable by untimed model checking, it is
reasonable to reset the global variable $timer$ when it reaches an
upperbound while increasing (see Line~\ref{l:timer} in
Algorithm~\ref{a:schedule}) in the model. Then $timer$ is of sort
\verb|Timer| and the upperbound would be the least common multiple of
the periods of all tasks.}

The global variable $timer$ is reset when it reaches an upper bound
while increasing, which is not shown in detail in
Algorithm~\ref{a:schedule} but is reasonable. The upper bound is the
least common multiple of the periods of all tasks. A sort \verb|Timer|
is defined:
\begin{verbatim}
  op [_/_] : Nat NzNat -> Timer [ctor] .
\end{verbatim}

\subsection{Modeling the System States}
The system can be considered as consisting of several parts: the tasks
which are scheduled, the scheduler itself, the hardware including
registers and stacks, and the interrupt source. The scheduler,
i.e. the function $schedule()$, can be described by a single variable
$timer$. We present the models of the other parts one by one.

\subsubsection{Tasks}
Each task is abstracted from its functionality as a \verb|Counter|.
Overhead for scheduling and switching is considered in our model. They
are treated as two system tasks. Every task is modeled as an object
instance of some subclass of the base class \verb|Task|:
\begin{verbatim}
  class Task | cnt : Counter .
  op error : -> Object [ctor] .
\end{verbatim}
where \verb|error| is an object indicating some task that misses its
deadline.

A periodic task, which is needed to be scheduled, is an object
instance of the subclass \verb|PTask| with additional attributes
\verb|priority|, \verb|period| and \verb|status|:
\begin{verbatim}
  class PTask | priority : Nat, period : NzNat, status : Status .
  subclass PTask < Task .
\end{verbatim}
where \verb|Status| is a sort with four constant constructors, same as
in the implementation:
\begin{verbatim}
  ops RUNNING INTERRUPT READY DORMANT : -> Status [ctor] .
\end{verbatim}
The list of periodic tasks, the variable $taskList$ in the
implementation, is modeled as an instance of the sort \verb|TaskList|,
which is a list of \verb|PTask|s\footnote{Following the Maude
  convention, variables would be written in capital letters.}:
\begin{verbatim}
  op null : -> TaskList [ctor] .
  op _::_ : Object TaskList ~> TaskList [ctor] .
  mb (< O:Oid : PTask |> :: L:TaskList) : TaskList .
  mb (error :: L:TaskList) : TaskList .
\end{verbatim}
Periodic tasks are identified by their indexes in the list.

On the other hand, a system task is an object instance of the subclass
\verb|SysTask| of \verb|Task| with no extra attributes:
\begin{verbatim}
  class SysTask .            subclass SysTask < Task .
\end{verbatim}
Different from periodic tasks, system tasks are organized in a
multiset of sort \verb|SysTasks|, and identified by their \verb|Oid|s.
%We abstract here from the detailed definitions thanks to the
%similarity.

\subsubsection{Hardware}
Our model considers two parts of hardware related to the interrupt
handling mechanism: the registers and the stack.

The set of registers is modeled as an object instance of the class
\verb|Regs|, with attributes \verb|pc| denoting the program counter,
\verb|mask| for the interrupt masking flag and \verb|ir| for the
interrupt request flag, respectively.
\begin{verbatim}
  class Regs | pc : TaskID, mask : Bool, ir : Bool .
\end{verbatim}
where the sort \verb|TaskID| is a supersort of \verb|MaybeNat| and
\verb|Oid|, referring to some task.
\begin{verbatim}
  subsorts MaybeNat Oid < TaskID .
\end{verbatim}
Some operations, such as \verb|getPc| and \verb|setMask|, are defined
on the class \verb|Regs|.

Then the hardware is naturally described by the sort \verb|Hardware|:
\begin{verbatim}
  op [_;_] : Object Stack ~> Hardware [ctor] .
  mb ([ < O:Oid : Regs |> ; S:Stack ]) : Hardware .
\end{verbatim}

\subsubsection{Interrupt Source}
The interrupt source is modeled as an object of class \verb|IntSrc|,
with attributes \verb|cycle| denoting the interrupt cycle $T$, and
\verb|val| for the value which will decrease from $T$ to $0$ while
time advances:
\begin{verbatim}
  class IntSrc | val : Time, cycle : Time .
\end{verbatim}

\subsubsection{System}
The entire system in our model is a composition of the parts
introduced above, with a sort \verb|System|\footnote{Some variable
  declarations are not shown for simplicity.}:
\begin{verbatim}
  op _____ : 
       TaskList Timer SysTasks Hardware Object ~> System [ctor] .
  mb (L T STS HW < O:Oid : IntSrc |>) : System .
\end{verbatim}

\subsection{Interrupt Requests}
\label{ss:ir}
Interrupt requests are performed by the source exactly every cycle
$T$, when the attribute \verb|val| decreases to \verb|zero|. The
requesting is an instantaneous action, thus is modeled by the
following instantaneous conditional rule applied on \verb|System|:
\begin{verbatim}
  crl [interrupt-request] :
    (L T STS HW ISRC) => (L T STS (HW).intReq reset(ISRC))
    if (ISRC).timeout .
\end{verbatim}
where the function \verb|_.timeout| examines whether the attribute
\verb|val| equals \verb|zero|, and \verb|_.intReq| sets the \verb|ir|
flag indicating there exists an interrupt request to be handled:
\begin{verbatim}
  op _.intReq : Hardware -> Hardware .
  eq [ REGS ; S ].intReq = [ (REGS).setIr ; S ] .
\end{verbatim}
Then the request will wait to be handled, which is explained in
Section~\ref{ss:inthandling}.


\subsection{Task Initiation}
\label{ss:init}
Periodic tasks are initiated sequentially by the function
$updateStatus()$ in Algorithm~\ref{a:schedule}, which is treated as an
instantaneous action in our model. It is modeled by the recursive
function as below:
\begin{verbatim}
  op updateStatus_with_ : TaskList Timer -> TaskList . 
  eq updateStatus null with TIMER = null .
  eq updateStatus (TASK :: L) with TIMER
       = (update TASK with TIMER) :: (updateStatus L with TIMER) .
\end{verbatim}
with \verb|TIMER| the current value of the global variable $timer$,
and function \verb|update_with_| updating the status of individual
task (Lines~\ref{l:startupdate}-\ref{l:endupdate} in
Algorithm~\ref{a:schedule}):
\begin{verbatim}
  op update_with_ : Object Timer ~> Object .
  ceq update < O : PTask | period : T, status : ST > with TIMER
        = if ST == DORMANT then < O : PTask | status : READY >
          else error fi
      if TIMER rem T == 0 .
  eq update < O : PTask | status : ST > with TIMER
       = if ST == RUNNING then < O : PTask | status : INTERRUPT >
         else < O : PTask |> fi [owise] .
\end{verbatim}
Given a task, if \verb|TIMER|($timer$) can be divided by its period,
this task should be initiated.  In the case where the task should be
initiated, it is set to be \verb|READY| if its \verb|status| is
\verb|DORMANT|; otherwise that means the previous job of this task is
not complete, hence it misses its deadline, producing an
\verb|error|. In the other case where the task should not be
initiated, its \verb|status| changes only if it is \verb|RUNNING|. 

We can see that \verb|updateStatus_with_| behaves the same as
$updateStatus()$.

\subsection{Interrupt Handling and Task Scheduling}
\label{ss:inthandling}
When an interrupt request occurs, it may not be detected immediately
by the system. It requires the \verb|mask| flag to be cleared. Once
the request is detected, it is handled by two steps: the interrupt
handling mechanism of the hardware (such as clearing \verb|ir|,
pushing context into stack and so on), and to invoke the $schedule()$
function. This behavior is modeled by the following instantaneous
rewrite rule:
\begin{verbatim}
  crl [interrupt-handle] :
    SYSTEM => ((SYSTEM).interrupt).startScheduling
    if (SYSTEM).existInt .
\end{verbatim}
where \verb|_.existInt| checks whether \verb|mask| is cleared
\emph{and} \verb|ir| is set. The function \verb|_.interrupt| models
the interrupt handling mechanism performed by the hardware and does
four things: (i) clearing the \verb|ir| flag, which means the request
has been handled; (ii) pushing the current \verb|pc| value into the
stack, storing the interrupted context; (iii) assigning
\verb|scheduling| of sort \verb|Oid| to \verb|pc|, which indicates
that the system is scheduling; and (iv) setting the \verb|mask| flag,
to mask coming interrupt requests. 

Unlike periodic tasks, even though the scheduling stage is modeled by
a \verb|Counter|, its functionality is too important to be abandoned.
We divide the behaviors of scheduling into three parts.  The first
part contains its timed behaviors. This part is modeled by regarding
scheduling as a system task of sort \verb|SysTask|. Modeling timed
behaviors of tasks is explained in Section~\ref{ss:timedbehavior}.
The other two parts together define its functionality. The second part
corresponds to Lines~\ref{l:updatestatus}-\ref{l:inc} in
Algorithm~\ref{a:schedule}. It performs updating the status of
$taskList$ and increasing $timer$ by $1$. This part is modeled by
function \verb|_.startScheduling| which applies instantaneously at the
beginning of scheduling, as shown in rule \verb|interrupt-handle|:
\begin{verbatim}
  op _.startScheduling : System -> System .
  eq (L T STS HW ISRC).startScheduling 
       = ((updateStatus L with T) inc(T) STS HW ISRC) .
\end{verbatim}
The third part corresponds to
Lines~\ref{l:startrun1st}-\ref{l:endrun1st}, finding the first
should-be-run periodic task and setting it to execute. It is modeled
by function \verb|_.finishScheduling|, which applies instantaneously
at the end of scheduling:
\begin{verbatim}
  op _.finishScheduling : System -> System .
  eq (L T STS HW ISRC).finishScheduling
       = (L T (finish scheduling in STS) HW ISRC).run1stTask .
\end{verbatim}
where \verb|finish_in_| resets the counter of task \verb|scheduling|,
and \verb|_.run1stTask| models
Lines~\ref{l:startrun1st}-\ref{l:endrun1st} in
Algorithm~\ref{a:schedule}, searching the task with highest priority
that has status \verb|INTERRUPT| or \verb|READY| then performing an
\emph{interrupt return} or executing it, respectively.

When the execution time of the system task \verb|scheduling| reaches
its computation requirement, the scheduling stage may finish. We model
this instantaneous action with the following rule:
\begin{verbatim}
  crl [scheduling-finish] :
    (L T STS HW ISRC) => (SYSTEM).finishScheduling
    if SYSTEM := (L T STS HW ISRC) 
       /\ (SYSTEM).running == scheduling 
       /\ scheduling mayFinish?in STS .
\end{verbatim}
where function \verb|_.running| returns the current \verb|pc| value of
the system, and \verb|_mayFinish?in_| checks whether the execution
time of the task reaches its computation requirement.
\hide{
\begin{verbatim}
  op _mayFinish?in_ : Oid SysTasks ~> Bool .
  eq O mayFinish?in [ < O : SysTask | cnt : C > REST ] 
       = C mayFinish? .
  op _mayFinish? : Counter -> Bool .
  eq [ R / [ MIN , MAX ] ] mayFinish?
       = if R lt MIN then false else true fi .
\end{verbatim}}

Similar to scheduling, the switching stage is divided into timed
behaviors of \verb|switching| and its functionality. \verb|switching|
starts when the execution time of the running periodic task reaches
its computation requirement, and finishes when its own execution time
does so. Two similar instantaneous rules are defined:
\begin{verbatim}
  crl [task-finish] :
    (L T STS HW ISRC) => (SYSTEM).startSwitching
    if SYSTEM := (L T STS HW ISRC)
       /\ some N := (SYSTEM).running
       /\ some N mayFinish?in L .
  crl [switching-finish] :
    (L T STS HW ISRC) => (SYSTEM).finishSwitching
    if SYSTEM := (L T STS HW ISRC)
       /\ (SYSTEM).running == switching
       /\ switching mayFinish?in STS .
\end{verbatim}

\subsection{Timed Behaviors of the System}
\label{ss:timedbehavior}
\emph{Timed behaviors} of the system consist of two parts: the
execution of tasks and the execution of the interrupt source. Both are
modeled together by the following single ``standard'' tick
rule~\cite{DBLP:journals/entcs/OlveczkyM07a}:
\begin{verbatim}
  crl [tick]:
    {SYSTEM} => {delta(SYSTEM, R)} in time R 
    if R le mte(SYSTEM) [nonexec] .
\end{verbatim}
where \verb|delta| defines the effects of time elapse on the system,
and \verb|mte| denotes the \emph{m}aximum amount of \emph{t}ime
allowed to \emph{e}lapse from the current state until an instantaneous
transition \emph{must} be performed. In fact, the core to model timed
behaviors is to define functions \verb|delta| and \verb|mte|. Notice
that the variable $R$ is \emph{continuous} with respect to the
specific time domain\footnote{Real-Time Maude contains built-in
  modules to define the time domain to be natural numbers and rational
  numbers, specifying \emph{discrete} time domains and \emph{dense}
  time domains, respectively.}  that we choose to instantiate our
model on, which is different from timed automata that discretize dense
time by defining ``clock region''.

Time affects the system by advancing both the running task whose $ID$
is loaded at \verb|pc| and the interrupt source simultaneously.  While
time elapses, \verb|cnt| of the former increases and \verb|val| of the
latter decreases, respectively:
\begin{verbatim}
  ceq delta((L T STS HW ISRC), R)
        = ((deltaTask(ID, L, R) T STS HW (deltaIS(ISRC, R)))
      if ID := (HW).getPc /\ ID :: MaybeNat .
\end{verbatim}
We omit details for the case where \verb|ID| is of sort \verb|Oid| due
to similarity. In that case, \verb|deltaTask| applies on \verb|STS|
instead of \verb|L|.

\verb|mte|, the maximum amount of time allowed to elapse, depends on
when the next instantaneous action must perform. Therefore, it is
decided by three arguments: the remaining time to finish the running
task, the remaining time to request the next interrupt, and whether or
not there exists an interrupt request detected for the moment:
\begin{verbatim}
  ceq mte(L T STS HW ISRC)
        = minimum( mteTask(ID, L),
                   minimum( mteIS(ISRC), mteIr(HW)))
      if ID := (HW).getPc /\ ID :: MaybeNat .
\end{verbatim}
where \verb|mteIr| returns \verb|zero| if there exists an interrupt
request detected in the system, or \verb|INF| which represents
\emph{infinity} otherwise. Again we do not show the case where
\verb|ID| is of sort \verb|Oid|, which is very similar. We should
point out that \verb|mteTask| computes the remaining time to reach the
maximum of the computation requirement of the task, since that is the
time at which a \verb|task-finish| transition \emph{must} happen.

\section{Formal Verification}
\label{s:verification}
In this section, we analyze our model of the RMS implementation within
different realistic scenarios.  Notice that from any (reasonable)
given initial state, the number of reachable states is finite, but may
be unknown, thanks to the upper bound given to $timer$, which provides
the potential for applying the untimed model checker.

\subsection{Properties}
The main objective is to verify the schedulability of a given set of
periodic tasks. To this end, we define an atomic proposition
\verb|taskTimeout| to hold if there exists an \verb|error| appearing
in the $taskList$ of the current state:
\begin{verbatim}
  op taskTimeout : -> Prop [ctor] .
  eq {L T STS HW ISRC} |= taskTimeout = containError(L) .
\end{verbatim}
where \verb|containError| returns \verb|true| if there is an object
\verb|error| existing in $L$. Then our desired
property--schedulability--can be formalized as the temporal logic
formula: \verb|[](~taskTimeout)|. As the property is not
\emph{clocked}, given an initial state \verb|init|, the following
untimed model checking command returns \verb|true| if the
schedulability property holds with no time limit; otherwise a trace
showing a counterexample is provided:
\begin{verbatim}
  (mc init |=u [](~taskTimeout) .)
\end{verbatim}

Another objective is to verify the correctness of the implementation,
in the sense that the periodic tasks are scheduled exactly with
respect to their priorities. The atomic proposition
\verb|scheduleCorrectly| is hence defined to hold if the running
periodic task is the one requested to be executed with the highest
priority:
\begin{verbatim}
  op scheduleCorrectly : -> Prop [ctor] .
  ceq {L T STS HW ISRC} |= scheduleCorrectly 
        = (not containError(L))
          and (if ID :: MaybeNat then shouldRun(ID, L)
               else true fi)
      if ID := (HW).getPc .
\end{verbatim}
where \verb|shouldRun(ID, L)| returns \verb|true| if the task
identified by \verb|ID|, probably \verb|none|, is the one possessing
the highest priority among those whose status is \verb|READY|,
\verb|RUNNING| or \verb|INTERRUPT|. Note that \verb|scheduleCorrectly|
requires no \verb|error| popping up, since our defined correctness
property makes no sense in the implementation once some task misses
its deadline. The correctness property is formalized as the temporal
logic formula: \verb|[]scheduleCorrectly|, and can be verified by the
following untimed model checking command provided an initial state
\verb|init|:
\begin{verbatim}
  (mc init |=u []scheduleCorrectly .)
\end{verbatim}

\subsection{Scenarios}
\label{ss:results}
We use the following setting for our verification, which is from the 
statistics provided by our industrial partner:
\begin{itemize}
\item The interrupt cycle is $5ms$.
\item The scheduling time ranges from $5{\mu}s$ to $9{\mu}s$, and the
switching time ranges from $2{\mu}s$ to $4{\mu}s$.
\item The initial state is with empty stack, empty \verb|pc|, cleared 
\verb|mask| and cleared \verb|ir|.
\end{itemize}

We have analyzed our model/implementation in ten different scenarios,
including both realistic ones provided by our industrial partner and
experimental ones designed by ourselves, four of them are described
below:
\begin{itemize}
\item Scenario (i) with two tasks: one having period $5ms$ and
  computation requirement $[2.4ms, 3ms]$, and the other having period
  $25ms$ and computation requirement $[6.1ms, 7ms]$.
\item Scenario (ii) with two tasks: one having period $5ms$ and
  computation requirement $[1.8ms, 2ms]$, and the other having period
  $25ms$ and computation requirement $[2.1ms, 2.3ms]$.
\item Scenario (iii) with three tasks: the first having period $5ms$
  and computation requirement $[2.5ms, 2.7ms]$, the second having
  period $10ms$ and computation requirement $[1.5ms, 2ms]$, and the
  third having period $25ms$ and computation requirement $[2.6ms,
    3ms]$.
\item Scenario (iv) with three tasks: the first having period $5ms$
  and computation requirement $[2.2ms, 2.5ms]$, the second having
  period $10ms$ and computation requirement $[1.4ms, 1.5ms]$, and the
  third having period $15ms$ and computation requirement $[4.3ms,
    4.5ms]$.
\end{itemize}

Instantiating our model on dense time domain and choosing the
\emph{maximal time sampling strategy}, the results of the model
checking show that both the schedulability and the correctness
properties hold in Scenarios (i-iii), but not in Scenario (iv). For
Scenario (iv), one counterexample of the schedulability is pictured in
Figure~\ref{f:counterexample}, where the third task (task$2$) misses
its deadline at the time $15ms$ that is marked by the tiny red
triangle. And the correctness fails in Scenario (iv) because of the
failure of schedulability.

\begin{figure}[h]
\begin{center}
\begin{picture}(300,110)(0,-10)
\thicklines
\put(0,0){\vector(1,0){320}}
\put(305,5){time(${\mu}s$)}
\thinlines
\put(0,0){\line(0,1){110}}
\put(-25,97){task$0$}
\put(-25,77){task$1$}
\put(-25,57){task$2$}
\put(-44,37){scheduling}
\put(-40,17){switching}
\put(-2,-8){$0$}

\dashline[30]{3}(0,100)(250,100)
\dashline[30]{3}(0,80)(280,80)
\dashline[30]{3}(0,60)(300,60)
\dashline[30]{3}(0,40)(210,40)
\dashline[30]{3}(0,20)(285,20)


\dashline[30]{3}(10,-10)(10,100)
\thicklines
\textcolor{red}{\put(0,40){\line(1,0){10}}}
\put(8,-18){$9$}

\thinlines
\dashline[30]{3}(50,0)(50,100)
\thicklines
\textcolor{blue}{\put(10,100){\line(1,0){40}}}
\put(35,-8){$2509$}

\thinlines
\dashline[30]{3}(55,-10)(55,80)
\thicklines
\textcolor{red}{\put(50,20){\line(1,0){5}}}
\put(46,-18){$2513$}

\thinlines
\dashline[30]{3}(80,0)(80,80)
\thicklines
\textcolor{blue}{\put(55,80){\line(1,0){25}}}
\put(65,-8){$4013$}

\thinlines
\dashline[30]{3}(85,-10)(85,60)
\thicklines
\textcolor{red}{\put(80,20){\line(1,0){5}}}
\put(76,-18){$4017$}

\thinlines
\dashline[30]{3}(100,0)(100,60)
\thicklines
\textcolor{blue}{\put(85,60){\line(1,0){15}}}
\put(90,-8){$5000$}

\thinlines
\dashline[30]{3}(110,-10)(110,100)
\thicklines
\textcolor{red}{\put(100,40){\line(1,0){10}}}
\put(101,-18){$5009$}

\thinlines
\dashline[30]{3}(150,0)(150,100)
\thicklines
\textcolor{blue}{\put(110,100){\line(1,0){40}}}
\put(135,-8){$7509$}

\thinlines
\dashline[30]{3}(155,-10)(155,60)
\thicklines
\textcolor{red}{\put(150,20){\line(1,0){5}}}
\put(146,-18){$7513$}

\thinlines
\dashline[30]{3}(200,0)(200,60)
\thicklines
\textcolor{blue}{\put(155,60){\line(1,0){45}}}
\put(185,-8){$10000$}

\thinlines
\dashline[30]{3}(210,-10)(210,100)
\thicklines
\textcolor{red}{\put(200,40){\line(1,0){10}}}
\put(198,-18){$10009$}

\thinlines
\dashline[30]{3}(250,0)(250,100)
\thicklines
\textcolor{blue}{\put(210,100){\line(1,0){40}}}
\put(230,-8){$12509$}

\thinlines
\dashline[30]{3}(255,-10)(255,80)
\thicklines
\textcolor{red}{\put(250,20){\line(1,0){5}}}
\put(243,-18){$12513$}

\thinlines
\dashline[30]{3}(280,0)(280,80)
\thicklines
\textcolor{blue}{\put(255,80){\line(1,0){25}}}
\put(260,-8){$14013$}

\thinlines
\dashline[30]{3}(285,-10)(285,60)
\thicklines
\textcolor{red}{\put(280,20){\line(1,0){5}}}
\put(272,-18){$14017$}

\thinlines
\dashline[30]{3}(300,0)(300,60)
\thicklines
\textcolor{blue}{\put(285,60){\line(1,0){15}}}
\put(289,-8){$15000$}

\textcolor{red}{\put(300,60){\vector(-1,0){0}}}

\end{picture}
\end{center}
\caption{An Counterexample of Scenario (iv).}
\label{f:counterexample}
\end{figure}

\subsection{Evaluation}
%\subsection{Soundness and Completeness of the Analysis}
We now show our results are both \emph{sound} and \emph{complete} in
this section.

An analysis method is called \emph{sound} if any counterexample found
using such a method is a real counterexample of the question, and
\emph{complete} if the fact that no counterexample can be found using
such a method implies no counterexample exists for the question in
analysis. The soundness of our results is easy to check, simply by
examining the counterexamples found. For instance, the counterexample
shown in Figure~\ref{f:counterexample} is a real counterexample,
implying that the result of Scenario~(iv) is sound. But this is not
the case for completeness, since we choose instantiating our model on
dense time domain to make it more real but giving rise to an infinite
state space which is unfeasible to exhaust.

In general, completeness of untimed model checking cannot be achieved
for any systems, any time sampling stategies and any
properties. However, \"Olveczky and Meseguer proved the completeness
of untimed temporal logic model checking, under the maximal time
sampling strategy, for a large class of real-time
systems~\cite{DBLP:journals/entcs/OlveczkyM07a}:
\begin{theorem}[\cite{DBLP:journals/entcs/OlveczkyM07a}]
\label{t:completeness}
Given a \emph{time-robust} real-time rewrite theory $\cal R$, a set
$P$ of \emph{tick-stabilizing} atomic propositions, an LTL formula
$\Phi$ (excluding the \emph{next} operator $\bigcirc$) whose atomic
propositions are contained in $P$. The untimed temporal logic model
checking verifying $\Phi$ is \emph{complete} under the maximal time
sampling strategy.
\end{theorem}

In order to prove the completeness of our verification, we need little
adjustments to our model.  For each task $\tau_i$, we collapse its
computation requirement $[C_i^{min},C_i^{max}]$ into the single value
$C_i^{max}$.  That is, in the adjusted model, we use the maximum time
that is possible to be required to complete the task, as in the
standard setting. The adjustment is reasonable for verifying our
properties, since the schedulability and correctness of the adjusted
model imply the schedulability and correctness of our original
model. In particular, this is because we can prove that, if a set of
tasks (including system ones) is schedulable when the computation
requirement of some task $\tau_i$ equals $C_i^{max}$, then the set
remains still schedulable when some jobs of $\tau_i$ become possible
to terminate in time $r\le C_i^{max}$. This statement is done by
proving Lemma~\ref{l:max}, and considering \verb|scheduling| and
\verb|switching| as tasks with same priority, which is higher than all
periodic tasks.
\begin{lemma}
\label{l:max}
Assume $r_s, r_f$ are the times at which a job of $\tau_i$ starts and
completes its execution, respectively. Let $S_i$ be the set of tasks,
which have priorities higher than $\tau_i$ or equal to it. Assumed
$r_s$ is fixed, $r_f$ reaches its maximum, if all job instances of
each task $\tau_j\in S_i$, which execute between $r_s$ and $r_f$, are
completed in time $C_j^{max}$.
\end{lemma}

We achieve the same verification results in the adjusted model as in
our original one: schedulability and correctness hold in
Scenarios~(i-iii)\footnote{In fact, positive results in the original
  model imply positive results in the adjusted one, since the set of
  behaviors of the adjusted model is a subset of those of the
  original.} but fail in Scenario~(iv).  The adjusted model is
time-robust and our defined atomic propositions--\verb|taskTimeout|
and \verb|scheduleCorrectly|--are tick-stabilizing, hence our analysis
using untimed model checking on the adjusted model is complete by
Theorem~\ref{t:completeness}. Therefore, the results shown in
Section~\ref{ss:results} for our original model are complete.


\hide{
To apply Theorem~\ref{t:completeness} on our analysis to show the
completeness, we only need to prove time-robustness of our model and
tick-stabilization of our defined atomic propositions,
\verb|taskTimeout| and \verb|scheduleCorrectly|. The latter is quite
trivial, while the former requires little adjustments to the model,
which collapse the possible finishing interval for some periodic task
into a single value that should be assigned to the possible maximum
computation time. Then we can apply Lemma~\ref{l:timerobustness} to
prove the time-robustness of our (adjusted) model.
\begin{lemma}
\label{l:timerobustness}
An object-oriented specification $\cal R$, with a conventional tick
rule, is \emph{time-robust} if the following conditions are satisfied
for all appropriate \emph{ground} terms $t$ and $r,r'$:

\noindent
(i) $mte(\delta(t,r))=mte(t)\dotdiv r$, for all $r\le mte(t)$;

\noindent
(ii) $\delta(t,0)=t$;

\noindent
(iii) $\delta(\delta(t,r),r')=\delta(t,r+r')$, for $r+r'\le mte(t)$;

\noindent
(iv) $mte(\sigma(l))=0$ for each ground instance $\sigma(l)$ of a
left-hand side of an instantaneous rewrite rule.
\end{lemma}

We achieve the same verification results for the adjusted model as for
the original one: schedulability and correctness hold in Scenarios
(i-iii) but fail in Scenario~(iv). The only thing remained to show the
completeness of our results is to prove the reasonability of the
adjustments, that is, if a set of tasks (including system ones) is
schedulable when the computation time of some task $\tau_i$ in it
equals to $C_i^{max}$, then when some jobs of $\tau_i$ become possible
to finish in time $r\le C_i^{max}$, the set of tasks remains still
schedulable. This is done by proving Lemma~\ref{l:max} and considering
\verb|scheduling| and \verb|switching| as having same and higher
priority than all periodic tasks.
\begin{lemma}
\label{l:max}
Assume the time $r_s$ at which a job of $\tau_i$ starts running is
fixed. The time $r_f$, at which this job finishes, reaches its maximum
if all job instances of tasks $\tau$'s, the priorities of which are
higher than or equal to that of $\tau_i$, running between $r_s$ and
$r_f$ finish in their maximum possible computation time.
\end{lemma}
}

\section{Related Work}
\label{s:relate}
\jx{This section should be adjusted w.r.t. the journal which it is
  submitted to.}  

In this section we discuss our results with related
work in three directions.

Considering schedulability test, Liu and
Layland~\cite{DBLP:journals/jacm/LiuL73} gave the famous sufficient
condition that a set of periodic tasks is schedulable with respect to
RMS if $\displaystyle \Sigma^n_{i=1} C_i/T_i \le n(2^{1/n}-1)$
holds. Then a more sufficient condition, known as \emph{Hyperbolic
  Bound}, which has the same complexity as Liu and Layland's one, was
proposed in~\cite{DBLP:journals/tc/BiniBB03}. On the other hand,
necessary and sufficient conditions for schedulability were derived
independently in~\cite{DBLP:journals/rts/SpruntSL89}
and~\cite{audsley1993deadline}, requiring more sophisticated analysis
on the task set. Nevertheless, all these results take no overhead into
account, being not as realistic as ours. Katcher et al. did consider
overhead in their schedulability analysis, under several models based
on different kinds of popular
implementations~\cite{DBLP:journals/tse/KatcherAS93}.  However, our
target implementation is not in their scope.  Furthermore, compared
with those theoretical analysis, our method using formal modeling and
verification has a great advantage that if our schedulability test
answers ``no'', it returns at the same time a real counterexample,
which is able to guide our engineer to adjust the design.

\cite{DBLP:conf/iceccs/CuiDT14} and~\cite{TianD2011} also made use of
model checking to analyze the RMS algorithm along the same line but
with different languages and tools. They considered only the ideal
setting as in~\cite{DBLP:journals/jacm/LiuL73}, but not an
implementation which contains much more details. Our model of the
implementation easily degenerates to a model of the RMS algorithm, if
we let the times for scheduling and switching be zero and let the
computation requirement of each task be a single value.

Finally, Maude and Real-Time Maude have been successfully applied on
large numbers of applications~\cite{DBLP:journals/jlp/Meseguer12},
especially on communication protocols, real-time and cyber-physical
systems. But few results are achieved on scheduling problems. RMS is
investigated using Real-Time Maude for the first time.

\section{Conclusion}
\label{s:conclusion}
We have formally modeled a realistic implementation of RMS algorithm
using Real-Time Maude, a modeling language based on rewriting
logic. By taking into acount the overhead of scheduling and switching,
assuming that the computation requirement for each task may not be
constant, we believe that our model contains enough details to be
analyzed for the behaviors of the real system. Two important
properties--schedulability and correctness--are verified by model
checking on our model, within several given scenarios. We also prove
at last the soundness and completeness of our results.


\bibliographystyle{splncs03} \bibliography{submission}
\end{document}

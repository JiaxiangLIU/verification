% This is LLNCS.DOC the documentation file of
% the LaTeX2e class from Springer-Verlag
% for Lecture Notes in Computer Science, version 2.4
\documentclass{llncs}
\usepackage{alltt}
\usepackage{amsmath}
%\usepackage{llncsdoc}
%
\begin{document}
%\thispagestyle{empty}
%\begin{flushleft}
%\end{flushleft}

\title{Formal Analysis of Rate-Monotonic Scheduling Implementation via Real-Time Maude}
\author{Jiaxiang Liu\inst{1,2}}
\institute{School of Software, Tsinghua University, Beijing, China
  \and \'Ecole Polytechnique, Palaiseau, France}
\maketitle
\thispagestyle{empty}

\section{Introduction}

\section{Real-Time Maude}

Real-Time Maude is a language and tool that extends Maude to support
the formal specification and analysis of real-time systems.

\subsection{Specification}

A Real-Time Maude module specifies a \emph{real-time rewrite theory}
$(\Sigma, E\cup A , IR, TR)$, where:

\begin{itemize}
\item $\Sigma$ is an algebraic \emph{signature}, that is, a set of
  declarations of \emph{sorts}, \emph{subsorts} and \emph{function
    symbols}.

\item $(\Sigma, E\cup A)$ is a \emph{membership equational logic
  theory}, with $E$ a set of possibly conditional equations, and $A$ a
  set of equational axioms such as associativiy, commutativity and
  identity.  $(\Sigma, E\cup A)$ specifies the system's state space as
  an algebraic data type, and includes a built-in specification of a
  sort \verb|Time|.

\item $IR$ is a set of \emph{labeled conditional rewrite rules}
  specifying the system's local transitions, each of which has the
  form $[l]~:~t\rightarrow t'\mbox{ \textbf{if}
  }\bigwedge^n_{j=1}cond_j$, where each $cond_j$ is an equality
  $u_j=v_j$, and $l$ is a \emph{label}. Such a rule specifies an
  \emph{instantaneous transition} from an instance of $t$ to the
  corresponding instance of $t'$, \emph{provided} the conditions hold.

\item $TR$ is a set of \emph{tick rules} $[l]~:~\{t\}\rightarrow\{t'\}
  \mbox{ \textbf{in time} }\tau\mbox{ \textbf{if} }cond$ that advance
  time in the \emph{entire} state $t$ by $\tau$ time units.
\end{itemize}

A class declaration $\texttt{class }C\texttt{ |
}att_1\texttt{:}s_1\texttt{,}\ldots\texttt{,}att_n\texttt{:}s_n$
declares a class $C$ with attributes $att_1$ to $att_n$ of sorts $s_1$
to $s_n$. An \emph{object} of class $C$ in a given state is
represented as a term $\texttt{< } O\texttt{:} C \texttt{ | }
att_1\texttt{:}val_1\texttt{,}\ldots\texttt{,}att_n\texttt{:}val_n\texttt{
  >}$ of sort \verb|Object|, where $O$, of sort \verb|Oid|, is the object's
\emph{identifier}, and where $val_1$ to $val_n$ are the current values of the
attributes $att_1$ to $att_n$. A \emph{subclass} inherits all the attributes
and rules of its superclasses.

\subsection{Formal Analysis}
In this 

\section{Rate-Monotonic Scheduling Algorithm and the Implementation}

\section{Formalizing the RMS Implementation}

\section{Formal Analysis}
\subsection{Properties}
\subsection{Completeness of the Analysis}

\section{Related Work}

\section{Conclusion}


\end{document}

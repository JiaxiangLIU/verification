\documentclass{beamer}
\usepackage[latin1]{inputenc}
\usepackage{hyperref}
\usepackage{colortbl}
\usetheme{Warsaw}
\title[Make a LaTeX presentation using Beamer]
      {Termination Analysis of Imperative Programs Using Bitvector Arithmetic}
\author{Stephan Falke\inst{1}, Deepak Kapur\inst{2}, and Carsten Sinz\inst{1}}
\institute{\inst{1} Institute for Theoretical Computer Science, KIT, Germany \and %
  \inst{2} Dept. of Computer Science, University of New Mexico, USA}
\date{\footnotesize Verified Software: Theories, Tools, Experiments \\ VSTTE 2012}

\AtBeginSubsection[]
{
  \begin{frame}<beamer>
    \frametitle{Layout}
    \tableofcontents[currentsection,currentsubsection]
  \end{frame}
}

\begin{document}

\begin{frame}
\titlepage
\end{frame}

\begin{frame}{Summary}
This paper proposes a novel method for \textcolor{red}{encoding the wrap-around
  behavior of bitvector arithmetic within integer arithmetic}, such
that existing methods for reasoning about the termination of integer
arithmetic programs can be employed for reasoning about the
termination of bitvector arithmetic programs.
\end{frame}
  

\section{Introduction}

\begin{frame}{Existing Methods}
  Bitvectors are treated as (unbounded) integers (or as real numbers).

  This approximation can cause errors in both directions:



%  \begin{itemize}
\begin{columns}
  \begin{column}{5cm}
    \begin{block}{Block title}
This is a block in blue
\end{block}
    
%  \item First column
\end{column}
\begin{column}{5cm}
%\item Second column
\end{column}
\end{columns}
 % \end{itemize}

\end{frame}

\subsection{sub1}
\begin{frame}{1st}
  List:
\begin{itemize}
\pause \item Beamer is a wonderful class
\pause \item One can make animations
\pause \item One uses the\textbf{pause} command, for example
\pause \item in order to bring in important ideas
\end{itemize}
\end{frame}

\subsection{sub2}
\begin{frame}{2}
  This is a short introduction to Beamer class.
  \begin{itemize}
\item<2-> appears from slide 2 on
\item<3-> appears from slide 3 on
\item<4-> appears from slide 4 on
\item<5-> appears from slide 5 on
\end{itemize}
\end{frame}


\section{TWO}
\begin{frame}{3}
  This is a short introduction to Beamer class.
  \begin{itemize}
\item<2-> appears from slide 2 on
\item<2-3> appears from slide 2 to slide 4
\item<4> appears on slide 4
\item<3-> appears from slide 3 on
\end{itemize}
\end{frame}

\begin{frame}{4}
  See
\begin{itemize}[<+->]
\item L
\item A
\item T
\item E
\item X
\end{itemize}
\end{frame}

\begin{frame}
\uncover<2->
{appear from slide 2 on\\}
\uncover<3-4>
{appears from 3 to slide 4\\}
\uncover<4>{appears on slide 4\\}
\uncover<3->{appears from slide 3 on\\}
\end{frame}

\begin{frame}

\only<2->
{appear from slide 2 on\\}
\only<3-4>
{appears from 3 to slide 4\\}
\only<4>{appear on slide 4\\}
\only<3->{appear from slide 2 on\\}

\end{frame}

\begin{frame}

\begin{itemize}
\item Language used by Beamer: L\uncover<2->{A}TEX
\item Language used by Beamer: L\only<2->{A}TEX
\end{itemize}

\end{frame}

\begin{frame}

\invisible<2>{This text will be invisible on slide 2, but not on others slides}\\
This text is always visible\\
\uncover<1->{Beamer} \uncover<2->{is}  \uncover<3->{super} \uncover<4->{powerful} 

\end{frame}


\begin{frame}{5}
  \alt<3>{I am on slide 3\\}{I am not on slide 3\\}
\only<2->
{appears from slide 2 on\\}
\only<3-4>
{appears from slide 3 to slide 4\\}
\only<4>{appears on slide 4\\}
\only<3->{appears from slide 3 on\\}
\alert<1>{This text} \alert<2>{is} \alert<3>{red}
\end{frame}

\begin{frame}{6}
\begin{itemize}
\item <+-| alert@+> Robert De Niro
\item <+-| alert@+> Brian De Palma
\item <+-| alert@+> Gerard Depardieu
\item <+-| alert@+> Tux
\end{itemize}
\end{frame}

\begin{frame}{7}
  Some colors ...\\
\color<2>{green}{Green color} \\
Great !!!
\end{frame}

\begin{frame}[label=link]
The link will point to this frame
\end{frame}

\begin{frame}{8}
  \hyperlink{link}{\beamergotobutton{Refer to this page}}

  \begin{block}{Block title}
This is a block in blue
\end{block}

\invisible<3->{
\begin{alertblock}{Alert-block title}
This is a block in red
\end{alertblock}}

\invisible<2>{
\begin{exampleblock}{Example-block title}
This is a block in green
\end{exampleblock}}
  \end{frame}

\begin{frame}{9}
\begin{tabular}{lcccc}
  Class & A & B & C & D \\\hline
  X     & 1 & 2 & 3 & 4 \\\pause
  Y     & 3 & 4 & 5 & 6 \pause\\
  Z     &5&6&7&8
\end{tabular}
\end{frame}

\begin{frame}{10}
\begin{tabular}{lc<{\onslide<2->}c<{\onslide<3->}c<{\onslide<4->}c<{\onslide}c}
  Class & A & B & C & D \\
  X     & 1 & 2 & 3 & 4 \\
  Y     & 3 & 4 & 5 & 6 \\
  Z     &5&6&7&8
\end{tabular}
\end{frame}

\begin{frame}{11}
\begin{columns}
\begin{column}{0.5\textwidth}
First column
\end{column}
\begin{column}{0.5\textwidth}
Second column
\end{column}
\end{columns}
\end{frame}


\end{document}
